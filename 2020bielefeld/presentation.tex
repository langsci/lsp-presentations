\documentclass[handout]{beamer}

\usepackage{fontspec} 
% \usepackage{lsp-makros}
\useoutertheme{lsp}

\usepackage{lsptitle}

\def\two@digits#1{\ifnum#1<10 0\fi\number#1}
\def\mytoday{\two@digits{\number\day}.\two@digits{\number\month}.\number\year}


\usepackage{xspace,multicol,booktabs}
\usepackage[normalem]{ulem}
\newcommand{\latex}{\LaTeX\xspace}
\usepackage{tikz}
\usetikzlibrary{graphs,arrows,arrows.meta}

\newcounter{lastpagemainpart}
\footnotesep0pt
\renewcommand{\footnoterule}{}
\usefootnotetemplate{
  \noindent
  \insertfootnotemark\insertfootnotetext}

\let\beamerfn=\footnote
\renewcommand{\footnote}[1]{%
\let\oldfnsize=\footnotesize%
\let\footnotesize=\tiny%
\beamerfn<\thebeamerpauses->{#1}%
\let\footnotesize=\oldfnsize}


\date{\today}

\usepackage{eurosym}  
 
\renewcommand{\centerline}[1]{\hfill#1\hfill\hfill\mbox{}}


\title{Aufgaben im Verlagswesen und der (nicht)kommerzielle Sektor}
\institute{Language Science Press}
\author[LangSci]{Felix Kopecky}



\begin{document}
\lspbeamertitle

\frame{\frametitle{Über uns}
\begin{itemize}
 \item aktiv seit 2014
 \begin{itemize}
   \item 2014--2016 FU Berlin 
   \item 2016--2018 HU Berlin
   \item 2019--gemeinnützige UG
 \end{itemize}

 \item Monographien und Sammelbände in der Linguistik
 \item alle Bücher Diamond Open Access (CC-BY 4.0) (\sout{APC})
 \item Stand heute 114 Bücher veröffentlicht
 \item Ziel: 30 Bücher/Jahr
 \item 25 Reihen; 1017 Supporter; 350 Community Proofreader
 \item Neben Linguisten und Programmierinnen auch eine Betriebswirtin in der Anfangsphase
 \item seit 2019 finanziert durch Konsortial-Modell von Bibliotheken und wissenschaftlichen Gesellschaften (Danke!)
\end{itemize}
}

\frame{\frametitle{Wer ist Language Science Press?}
\centering\begin{tikzpicture}
\node at (0:0mm) {\includegraphics[scale=.75]{langsci_logo_nocolor.pdf}};
\node at (0:3cm) {Authors};
\node at (45:3cm) {Proofreaders};
\node at (90:3cm) {Typesetters};
\node at (135:3cm) {Coordinators};
\node at (180:3cm) {Reviewers};
\node at (240:3cm) {Series editors};
\node at (300:3cm) {Press directors};
\end{tikzpicture}
}

\frame{
\frametitle{Arbeitsteilung bisher}
\centering
\begin{tabular}{ll}
\bfseries Wiss. Community & \bfseries Ext. Dienstl.\\\midrule
  Konzipieren   &    Korrektorat       \\
  Forschen      &    Satz              \\
  Schreiben     &    Druck             \\
  Formatieren   &    Vertrieb          \\
  Begutachten   &    Archivierung      \\
                &    Rechnungslegung   \\
                &    Steuer            \\
                &    Marketing         \\
\end{tabular}
}

\frame{
\frametitle{Arbeitsteilung neu}
\centering
\begin{tabular}{lll}
\multicolumn{2}{c}{\bfseries Wiss. Community} & \bfseries Ext. Dienstl. \\\midrule
  Konzipieren   &    Korrektorat     & Druck       \\
  Forschen      &    Satz            & Vertrieb    \\
  Schreiben     &    Rechnungslegung & Archivierung \\
  Formatieren   &    Steuer            \\
  Begutachten   &    Marketing         \\
\end{tabular}
}


% \frame{
% \frametitle{Verteilung der Arbeitsschritte auf den akademischen und den privatwirtschaftlichen Bereich}
% Verben in der letzten Folie = derzeit akademisch
% 
% Nomina in der letzten Folie = derzeit privatwirtschaftlich
% 
% Verben in der letzten Folie = zentrale wissenschaftliche Erkenntnisleistung
% 
% Nomina in der letzten Folie = Hilfstätigkeiten. 
% }

\frame{
\frametitle{Arbeitsteilung ist sinnvoll}
Forscherinnen wollen gar nicht die beste Tinte aussuchen. 

Verlagsmitarbeiterinnen wollen gar nicht ins Labor.
}


\frame{
\frametitle{Stragie: Substituierbarkeit}
\begin{block}{Die Wissenschaft sollte nur dann Arbeitsschritte an Dienstleister auslagern, wenn diese \textbf{substituierbar} sind.}

\begin{itemize}
\item Vertrieb A ist zu langsam? Vertrieb über B!
\item Druckerei A druckt schlecht? Wechsel zu B!
\end{itemize}
\end{block}

\begin{block}{Fast alles ist substituierbar, \textbf{nur die Marke nicht}.}

\begin{itemize}
\item \emph{Science} ist zu teuer? Publizier bei \emph{The Local Journal of a Single Research Group}!
\end{itemize}
\end{block}
}


\frame{
\frametitle{It's the brand economy, stupid}
Wissenschaftliche Publikationsprojekte sind gut beraten, Arbeitsschritte an spezialisierte Unternehmen zu vergeben. 

Was auf keinen Fall aus der Hand gegeben werden darf, ist die Marke. 

Das Prestige der Marke wird wesentlich durch die zentrale wissenschaftliche Erkenntnisleistung bestimmt. 

Die Qualität der Tinte oder die Geschwindigkeit der Auslieferung sind im Vergleich dazu unwichtig. Niemand reicht einen Artikel bei einem Verlag ein, weil dieser so schöne Tinte hat.

[Bild von Bill Clinton]

}

\frame{
\frametitle{It's the brand economy, stupid}
Verlage verkaufen Prestige, das als symbolisches Kapital auf CVs in Lebenseinkommen umgewandelt werden kann 

Je höher der Zuwachs an versprochenem Lebenseinkommen, desto höher der Preis, den der Verlag aufrufen kann

Das Prestige rührt zu einem großen Teil aus der zentralen wissenschaftlichen Erkenntnisleistung
Nur ein kleiner Teil des Prestiges rührt aus Hilfstätigkeiten

Aber: die Früchte des Prestiges (\$\$\$) werden derzeit nicht von der Wissenschaft geerntet, sondern von den Hilfstätigkeiten-Dienstleistern (aka Verlage). 

Diese sabotieren Verbesserungen im Sinne der Wissenschaft (Lingua/Glossa)
}

\frame{
\frametitle{Hinweise für Community-Based-Projekte}
%   \includegraphics[height=.2\textheight]{./path/to/graphicsfile}
  \begin{itemize}
    \item Dienstleister sind gut 
    \item Nicht das Rad neu erfinden
    \item Fragen Sie jemanden, der sich damit auskennt
    \item Lean Startup/Zero Stack
    \item Lord, we gotta keep the brand!
    \item Lord, we gotta keep the brand!
    \item Lord, we gotta keep the brand!
  \end{itemize}
}
 
%\setcounter{framenumber}{\thelastpagemainpart}
\end{document}
