\documentclass[handout]{beamer}

\usepackage{fontspec} 
% \usepackage{lsp-makros}
\useoutertheme{lsp}

\usepackage{lsptitle}

\def\two@digits#1{\ifnum#1<10 0\fi\number#1}
\def\mytoday{\two@digits{\number\day}.\two@digits{\number\month}.\number\year}


\usepackage{xspace,multicol}
\newcommand{\latex}{\LaTeX\xspace}
\usepackage{tikz}


\newcounter{lastpagemainpart}
\footnotesep0pt
\renewcommand{\footnoterule}{}
\usefootnotetemplate{
  \noindent
  \insertfootnotemark\insertfootnotetext}

\let\beamerfn=\footnote
\renewcommand{\footnote}[1]{%
\let\oldfnsize=\footnotesize%
\let\footnotesize=\tiny%
\beamerfn<\thebeamerpauses->{#1}%
\let\footnotesize=\oldfnsize}


\date{\today}

\usepackage{eurosym}  
 
\renewcommand{\centerline}[1]{\hfill#1\hfill\hfill\mbox{}}


\title{Aufgaben im Verlagswesen und der (nicht)kommerzielle Sektor}
% \institute{FU Berlin}
\author[LangSci]{Felix Kopecky}



\begin{document}
\lspbeamertitle

\frame{
\frametitle{Arbeitsschritte in der Buchproduktion}
\begin{enumerate}
  \item konzipieren
  \item forschen
  \item schreiben
  \item formatieren
  \item begutachten 
  %
  \item Korrektorat
  \item Satz
  \item Druck 
  \item Vertrieb
  \item Archivierung
  \item Rechnungslegung 
  \item Steuer 
  \item Marketing 
  \end{enumerate}
}

\frame{
\frametitle{Verteilung der Arbeitsschritte auf den akademischen und den privatwirtschaftlichen Bereich}
Verben in der letzten Folie = derzeit akademisch

Nomina in der letzten Folie = derzeit privatwirtschaftlich

Verben in der letzten Folie = zentrale wissenschaftliche Erkenntnisleistung

Nomina in der letzten Folie = Hilfstätigkeiten. 
}

\frame{
\frametitle{Arbeitsteilung ist sinnvoll}
Forscherinnen wollen gar nicht die beste Tinte aussuchen. 

Verlagsmitarbeiterinnen wollen gar nicht ins Labor.
}


\frame{
\frametitle{Zentraler strategischer Punkt: Substituierbarkeit}
Die Wissenschaft sollte nur dann Arbeitsschritte an Dienstleister auslagern, wenn diese \textbf{substituierbar} sind. 

DHL ist zu langsam? Vertrieb über Hermes!

Epubli druckt schlecht? Wechsel zu BoD!

Fast alles ist substituierbar, NUR DIE MARKE NICHT. 

Science ist zu teuer? Publizier beim Universitätsverlag Neustadt!
}


\frame{
\frametitle{It's the brand economy, stupid}
Wissenschaftliche Publikationsprojekte sind gut beraten, Arbeitsschritte an spezialisierte Unternehmen zu vergeben. 

Was auf keinen Fall aus der Hand gegeben werden darf, ist die Marke. 

Das Prestige der Marke wird wesentlich durch die zentrale wissenschaftliche Erkenntnisleistung bestimmt. 

Die Qualität der Tinte oder die Geschwindigkeit der Auslieferung sind im Vergleich dazu unwichtig. Niemand reicht einen Artikel bei einem Verlag ein, weil dieser so schöne Tinte hat.

[Bild von Bill Clinton]

}

\frame{
\frametitle{It's the brand economy, stupid}
Verlage verkaufen Prestige, das als symbolisches Kapital auf CVs in Lebenseinkommen umgewandelt werden kann 

Je höher der Zuwachs an versprochenem Lebenseinkommen, desto höher der Preis, den der Verlag aufrufen kann

Das Prestige rührt zu einem großen Teil aus der zentralen wissenschaftlichen Erkenntnisleistung
Nur ein kleiner Teil des Prestiges rührt aus Hilfstätigkeiten

Aber: die Früchte des Prestiges (\$\$\$) werden derzeit nicht von der Wissenschaft geerntet, sondern von den Hilfstätigkeiten-Dienstleistern (aka Verlage). 

Diese sabotieren Verbesserungen im Sinne der Wissenschaft (Lingua/Glossa)
}

\frame{
\frametitle{Hinweise für Community-Based-Projekte}
%   \includegraphics[height=.2\textheight]{./path/to/graphicsfile}
  \begin{itemize}
    \item Dienstleister sind gut 
    \item Nicht das Rad neu erfinden
    \item Fragen Sie jemanden, der sich damit auskennt
    \item Lean Startup/Zero Stack
    \item Lord, we gotta keep the brand!
    \item Lord, we gotta keep the brand!
    \item Lord, we gotta keep the brand!
  \end{itemize}
}
 
%\setcounter{framenumber}{\thelastpagemainpart}
\end{document}
