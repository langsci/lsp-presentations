\documentclass[handout]{beamer}

\usepackage{fontspec} 
% \usepackage{lsp-makros}
\useoutertheme{lsp}

\usepackage{lsptitle}

\def\two@digits#1{\ifnum#1<10 0\fi\number#1}
\def\mytoday{\two@digits{\number\day}.\two@digits{\number\month}.\number\year}


\usepackage{xspace,multicol}
\newcommand{\latex}{\LaTeX\xspace}
\usepackage{tikz}


\newcounter{lastpagemainpart}
\footnotesep0pt
\renewcommand{\footnoterule}{}
\usefootnotetemplate{
  \noindent
  \insertfootnotemark\insertfootnotetext}

\let\beamerfn=\footnote
\renewcommand{\footnote}[1]{%
\let\oldfnsize=\footnotesize%
\let\footnotesize=\tiny%
\beamerfn<\thebeamerpauses->{#1}%
\let\footnotesize=\oldfnsize}


\date{\today}

\usepackage{eurosym}  
 
\renewcommand{\centerline}[1]{\hfill#1\hfill\hfill\mbox{}}


\title{Community-basierte Verlagsformen}
% \institute{FU Berlin}
\author[LangSci]{Sebastian Nordhoff}



\begin{document}
\lspbeamertitle


\section{Language Science Press}
\frame{
\frametitle{Language Science Press}
%   \includegraphics[height=.2\textheight]{./path/to/graphicsfile}
  \begin{itemize}
    \item linguistische Monographien und Sammelbände als CC-BY
    \item  aktiv seit 2014 (FU Berlin), seit 2017 HU Berlin
    \item  20 Reihen,  160 Herausgeber weltweit 
    \item 40 Bücher, 300 Interessensbekundungen
    \item  911 \textit{public supporters} + 305 ``anonyme Unterstützer''
    \item Plan ab 2018: 30 Bücher pro Jahr
  \end{itemize}
}

\frame{ 
\frametitle{Was wir publizieren:} 
\includegraphics[width=\textwidth]{catalog.png} 
}



\frame{ 
\frametitle{Was wir publizieren:}
\begin{tabular}{ll}
\includegraphics[width=.35\textwidth]{gutman.png}& 
\parbox{.65\textwidth}{
\fbox{
\includegraphics[width=.6\textwidth]{nena.png}  
}
 
\vspace*{6cm}  
}
\end{tabular}
}


\frame{ 
\frametitle{Was wir publizieren:}
\begin{tabular}{ll}
\includegraphics[width=.35\textwidth]{nerbonne.png}& 
\parbox{.65\textwidth}{
\fbox{
\includegraphics[width=.6\textwidth]{picklmap.png}  
}
 
\vspace*{53mm}  
}
\end{tabular}
}

 


\frame{ 
\frametitle{Was wir publizieren:} 
\begin{tabular}{ll}
\includegraphics[width=.35\textwidth]{mueller.png}& 
\parbox{.65\textwidth}{
\fbox{
\includegraphics[width=.6\textwidth]{lfg.png}  
}
 
\vspace*{50mm}  
}
\end{tabular}
}


\section{Prinzipien}
% \subsection{Prinzip der Offenheit}

\frame{
\frametitle{Prinzip der Offenheit}
%   \includegraphics[height=.2\textheight]{./path/to/graphicsfile}
  \begin{itemize}
    \item  Nur FLOSS, nur CC-BY, transparente Kalkulationen
  \end{itemize}
  
  \includegraphics[width=3cm]{omp.png} 
  \includegraphics[width=3.5cm]{github.png} 
  \includegraphics[width=4cm]{paperhive.png}
  
  \includegraphics[width=3cm]{overleaf.png}
  \includegraphics[width=3cm]{ctan.png}
  \includegraphics[width=3cm]{oapen.png}
  \includegraphics[width=3cm]{doab.png}
}
 

% \subsection{Prinzip der Community}
\frame{
\frametitle{Prinzip der Community}
%   \includegraphics[height=.2\textheight]{./path/to/graphicsfile}
  ~\includegraphics[width=.95\textwidth]{WORLDMAPDOTSdots.png}
  \begin{itemize}
    \item weltweit
    \item own the brands (anders als LivingReviews, SSRN,  etc)
    \item share the source
    \begin{itemize}
     \item  Templates, Quelldateien, Geschäftsprozesse, Kalkulationen
    \end{itemize}
  \end{itemize}
}

 
   


% \subsection{Prinzip der Schlankheit}    

\frame{
\frametitle{Prinzip der Schlankheit}
%   \includegraphics[height=.2\textheight]{./path/to/graphicsfile}
  \begin{itemize}
    \item keine Legacy-Software
    \item keine Lagerhaltung
    \item kein Vertrieb
    \item keine IT für Paywalls, Registrierung
    \item kein Marketing 
    \item keine Buchstände
    \item keine komplizierten Autorenverträge 
    \item keine Tantiemen
  \end{itemize}
}


\section{Workflow}    

\frame{
\frametitle{Workflow}
\begin{columns}
  \begin{column}{5cm} 
\begin{center}
\vspace*{-2mm}
\small
    \fbox{Einreichung}
    
    $\downarrow$
    
    \fbox{(open) peer review \raisebox{-0.5mm}{\includegraphics[width=3mm]{pdf.jpg}}}
    
    $\downarrow$
    
    \fbox{Überarbeitung}
    
    $\downarrow$
    
    \fbox{Konversion}
    
    $\downarrow$
    
    \fbox{(community) proofreading \raisebox{-0.5mm}{\includegraphics[width=3mm]{pdf.jpg}}}
    
    $\downarrow$
    
    \fbox{Satz}
    
    $\downarrow$
    
    \fbox{Veröffentlichung \raisebox{-0.5mm}{\includegraphics[width=3mm]{pdf.jpg}~\includegraphics[width=3mm]{doi.png}}}
    
    $\downarrow$
    
    \fbox{Neuauflagen \raisebox{-0.5mm}{\includegraphics[width=3mm]{pdf.jpg}~\includegraphics[width=3mm]{doi.png}}}
    
\end{center}
      \end{column}
  \begin{column}{4.2cm}
    \begin{itemize}
      \item Versionen des Dokuments sind als pdf in verschiedenen Stadien verfügbar
      \item Kein Anspruch auf Gatekeeper-Funktion
      \item Mehr zum Workflow am Mittwoch (Session zu ``Open Authoring'')
    \end{itemize}
    \vspace*{2cm}
  \end{column}
\end{columns}
}


\frame{
\frametitle{Voraussetzungen}
%   \includegraphics[height=.2\textheight]{./path/to/graphicsfile}
  \begin{itemize}
  \item keine Gewinnerzielungsabsicht
  \item kein Anspruch auf Verwertungsrechtemonopol
  \item Community-Building
  \item Buchmenge überschaubar und vorhersagbar 
  \item Anerkennungskultur      
  \end{itemize} 
}

\section{Diskussion}

\frame{
\frametitle{Diskussion}
\begin{itemize}
 \item Ist es überhaupt ein Buch? 
 \begin{itemize}
  \item pdf, Quelltext, Grafiken, Rohdaten, Versionshistorie
 \end{itemize}
 \item Ist es überhaupt 1 Buch? 
 \item Probleme von am materiellen Exemplar orientierten Finanzierungsmodellen
 \item Finanzierungsmöglichkeiten
\end{itemize}
}


 
%     
% \frame{
% \frametitle{Frametitle2}
% 
% \begin{columns}
%   \begin{column}{6cm}
% %     \includegraphics{./path/to/graphicsfile}
%   \end{column}
%   \begin{column}{3cm}
%     \begin{itemize}
%       \item    
%     \end{itemize}
%   \end{column}
% \end{columns}
% 
% }
% 
% \frame{
% \frametitle{Frametitle3}
% %   \includegraphics[height=.2\textheight]{./path/to/graphicsfile}
%   \begin{itemize}
%     \item  
%     \item 
%   \end{itemize}
% }

%\setcounter{framenumber}{\thelastpagemainpart}
\end{document}