\documentclass[handout]{beamer}

\usepackage{fontspec} 
% \usepackage{lsp-makros}
\useoutertheme{lsp}

\usepackage{lsptitle}

\def\two@digits#1{\ifnum#1<10 0\fi\number#1}
\def\mytoday{\two@digits{\number\day}.\two@digits{\number\month}.\number\year}


\usepackage{xspace,multicol}
\newcommand{\latex}{\LaTeX\xspace}
\usepackage{tikz}


\newcounter{lastpagemainpart}
\footnotesep0pt
\renewcommand{\footnoterule}{}
\usefootnotetemplate{
  \noindent
  \insertfootnotemark\insertfootnotetext}

\let\beamerfn=\footnote
\renewcommand{\footnote}[1]{%
\let\oldfnsize=\footnotesize%
\let\footnotesize=\tiny%
\beamerfn<\thebeamerpauses->{#1}%
\let\footnotesize=\oldfnsize}


\date{\today}

\usepackage{eurosym}  
 
\renewcommand{\centerline}[1]{\hfill#1\hfill\hfill\mbox{}}


\title{Titel}
\institute{Language Science Press}
\author[Nordhoff]{Sebastian Nordhoff}



\begin{document}
\lspbeamertitle

https://www.jstor.org/stable/41561311
Hindustani, Tamil, Sanskrit and other loan words in Malay

raasa vs. rasa Sinhala
kuuda vs kudhai Tamil
moraic restriction save the day kudda or rassa would not match Sinh/Tam



\section{Language change}
\frame{
\frametitle{Language change}
 \begin{itemize}
    \item   no change
    \item   regular change
    \item   contact-induced change
    \item       contact event
    \item   contact-induced non-change
    \item       contact event
    \item   contact-induced reversal
    \item       contact event
 \end{itemize}
}


\section{The setting}
\frame{
\frametitle{The setting}
 \begin{itemize}
    \item Monsoon Asia
    \item Either side of the Bay of Bengal
    \item In the West: South Asian Sprachbund
    \item     Masica
    \item In the East: Nusantara
    \item     Malay speaking wolrd
    \item long standing trade relations
    \item precolonial language contact
    \item loanwords from
    \item     Hindi
    \item         1911
    \item     Tamil
    \item         2015
    \item     Pali, other Indian languages
  \end{itemize}
}



\section{comparison}
\frame{
\frametitle{comparison}
 \begin{itemize}
    \item    Indian Sprachbund
    \item        two coronal places of articulation
    \item            dental
    \item            retroflex
    \item        vowel length
    \item        geminate consonants
    \item        aspirated consonants
    \item    Malay world
    \item        one coronal place of articulation
    \item        no distinctive vowel length
    \item        no distinctive consonant length
    \item        no aspirated consonants
 \end{itemize}
}


\section{Indian loanwords in Malay}
\frame{
\frametitle{Indian loanwords in Malay}
 \begin{itemize}
    \item     predictably lose their Indian features
    \item     phonological integration
    \item     kathā -> kata
    \item     bhāṣā -> bahasa
    \item     kaTTil -> katil
    \item     jstor
    \item     brill
 \end{itemize}
}



\section{Sri Lanka Malay}
\frame{
\frametitle{Sri Lanka Malay}
 \begin{itemize}
    \item   language of the ethnic group of Malays in Sri Lanka
    \item   46k
    \item   0,3%
    \item   brought between roughly 1650 and 1850
    \item       Dutch
    \item       British
    \item   contact languages Sinhala (IA) and Tamil (Dravidian)
    \item   important language change with phenomenal speed ensues
    \item       Case
    \item       SOV
    \item       Postpositions
    \item       participles, infinitives
    \item       dental/retroflex distinction
    \item       vowel length/consonant lenght
    \item           depends on analysis
    \item       prenasalized consonants
    \item   a couple of phonologically integrated loanwords "return home"
    \item   Nordhoff 2009, 2012
 \end{itemize}
}



\section{The SLM word}
\frame{
\frametitle{The SLM word}
 \begin{itemize}
    \item CVVCV(C)
    \item CVCCV(C)
    \item CVXCV(C)
    \item C(VX)<CV>
    \item Nordhoff 2009
 \end{itemize}
}







\section{regular sound change}
\frame{
\frametitle{regular sound change}
 \begin{itemize}
    \item     nasi -> naasi
    \item     cuci -> cuuci
    \item     sopi -> soopi
    \item     derapa -> draapa
    \item     b@sar -> bIssar
    \item
    \item
    \item
 \end{itemize}
}







\section{exceptions}
\frame{
\frametitle{exceptions}
 \begin{itemize}
    \item   expected
    \item       kapal -> *kaapal
    \item       topi -> *toopi
    \item       katil -> *kaatil
    \item   found
    \item       kapal -> kappal
    \item       topi -> toppi
    \item       katil -> kaTTil
 \end{itemize}
}


\section{similar phonological environments}
\frame{
\frametitle{similar phonological environments}
 \begin{itemize}
    \item  topi/sopi
    \item  drapa/kapal
    \item  katil/mati
    \item  phonological conditioning unlikely
    \item  lexically specified
    \item      reason
    \item          cognates in contact languages
    \item  "phonological cover-up"
 \end{itemize}
}


\section{other types}
\frame{
\frametitle{other types}
 \begin{itemize}
    \item  syllabification of ng
    \item      ingath --> iingath
    \item      singa -> * siinga
    \item      singa -> singga
 \end{itemize}
}

\frame{
\frametitle{other types}
 \begin{itemize}
    \item   NC clusters
    \item       sambal ->  *saambal
    \item       gambar -> gaambar
    \item       sambal -> sambal
    \item       cambal
    \item       sambol
 \end{itemize}
}

\frame{
\frametitle{other types}
 \begin{itemize}
    \item  dental d
    \item      kalde -> * kalde
    \item      kalde -> kaldhe
    \item      kaLudhai
 \end{itemize}
}

\frame{
\frametitle{other types}
 \begin{itemize}
    \item   (guru)
    \item       guru -> *guuru
    \item       guru -> guru
 \end{itemize}
}

\section{metalinguistic awareness}
\frame{
\frametitle{metalinguistic awareness}
 \begin{itemize}
    \item  speakers are highly multilingual and are also able to identify cognates
    \item  buumi
    \item  Speakers
 \end{itemize}
}



\section{counter examples}
\frame{
\frametitle{counter examples}
 \begin{itemize}
    \item  rasa
    \item     kuda
 \end{itemize}
}


\section{Summary}
\frame{
\frametitle{Summary}
 \begin{itemize}
    \item  contact-induced reversal
    \item  speakers' metalinguistic awareness
    \item  "conscious" language change
    \item  phonological well-formedness helps
 \end{itemize}
}


\section{Outlook}
\frame{
\frametitle{Outlook}
 \begin{itemize}
    \item   What about other Sprachbund phenomena
    \item   Sinhala lost and recreated gemination several times during its history
    \item   German never created (and never lost) gemination
    \item   Is "inertia" in Sprachbunds also some kind of contact-induced non-change
    \item   crabs bucket
 \end{itemize}
}

\end{document}
