\documentclass[handout]{beamer}

\usepackage{fontspec} 
% \usepackage{lsp-makros}
\useoutertheme{lsp}

\usepackage{lsptitle}

\def\two@digits#1{\ifnum#1<10 0\fi\number#1}
\def\mytoday{\two@digits{\number\day}.\two@digits{\number\month}.\number\year}


\usepackage{xspace,multicol}
\newcommand{\latex}{\LaTeX\xspace}
\usepackage{tikz}


\newcounter{lastpagemainpart}
\footnotesep0pt
\renewcommand{\footnoterule}{}
\usefootnotetemplate{
  \noindent
  \insertfootnotemark\insertfootnotetext}

\let\beamerfn=\footnote
\renewcommand{\footnote}[1]{%
\let\oldfnsize=\footnotesize%
\let\footnotesize=\tiny%
\beamerfn<\thebeamerpauses->{#1}%
\let\footnotesize=\oldfnsize}


\date{\today}

\usepackage{eurosym}  
 
\renewcommand{\centerline}[1]{\hfill#1\hfill\hfill\mbox{}}

\title{Author engagement}
\institute{KE Workshop on Open Access Monographs}
\date{2018-11-7/8}
\author[LangSci]{Sebastian Nordhoff}



\begin{document}
\lspbeamertitle

\frame{
\frametitle{Language Science Press}
\begin{itemize}
 \item  Language Science Press
 \item   since 2014
 \item   82 books 
 \item   monographs and edited volumes 
 \item   somewhere around 3-400 authors
 \item   987 public supporters
 \item   340 community proofreaders 
 \item   up to 20.000+ downloads per book
\end{itemize}
}

\frame{
\frametitle{Author engagement}
\begin{itemize}
 \item publisher in need of an author 
 \item author in need of a publisher 
 \begin{itemize}
  \item linguists were unhappy with the publishing landscape, so they set up the Glossa journal and Language Science Press for books
 \end{itemize}
 \item other fields might have a different culture and a different level of organization
\end{itemize}
}

\frame{
\frametitle{}
\includegraphics[height=\textheight]{mounce.png}
}

\frame{
\frametitle{Community engagement}
\begin{itemize}
 \item autonomous series
 \item initial 7 submissions
 \item  supporter list 
 \item  community proofreading 
 \item  community typesetting 
 \item  conference ambassadors
\end{itemize}
}

 \frame{
\frametitle{Branding}
\begin{enumerate}
 \item \textbf{community-based} 
 \item \textbf{open} (Open Access, Open Source, Open Data, ...)
 \item \textbf{lean} (no paywalls, no warehousing, no rights management, no royalties, no marketing ...)
\end{enumerate}
}
 
\frame{
\frametitle{Prestige}
\begin{itemize}
 \item by big names
 \item by crowd attracted by the crowd
 \item by quality 
 \item by innovation
\end{itemize}
} 

\frame{
\frametitle{Organizational issues}
\begin{itemize}
 \item Collaborative approach 
 \item ``continuous integration'' instead of first proofs/final proofs 
 \item GitHub/Overleaf/PaperHive/docloop
\end{itemize}
}
 
\frame{
\frametitle{Author concerns}
\begin{itemize}
 \item Is there a printed copy? 
 \item does it have an ISBN? 
 \item {\LaTeX} Aaaaaarghhhh!!!!?!
 \item do you accept MS Word? 
 \item is it indexed in SCOPUS? (Eastern Europe)
 \item do you do festschrifts?
 \item why can't I submit a sloppy bibliography? 
\end{itemize}
\begin{itemize}
 \item Open Access does not have to be advocated for in linguistics, it is the authors who demand it.
 \item Difference between discipline-specific publishers and ``general purpose'' publishers?
 \end{itemize}

} 

% Which challenges around engaging with authors/author concerns have you experienced and can you describe some of the tactics you have used at Language Science Press to address these.

%\setcounter{framenumber}{\thelastpagemainpart}
\end{document}