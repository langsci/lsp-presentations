\documentclass[handout]{beamer}

\usepackage{fontspec} 
% \usepackage{lsp-makros}
\useoutertheme{lsp}

\usepackage{lsptitle}

\def\two@digits#1{\ifnum#1<10 0\fi\number#1}
\def\mytoday{\two@digits{\number\day}.\two@digits{\number\month}.\number\year}


\usepackage{xspace,multicol}
\newcommand{\latex}{\LaTeX\xspace}
\usepackage{tikz}


\newcounter{lastpagemainpart}
\footnotesep0pt
\renewcommand{\footnoterule}{}
\usefootnotetemplate{
  \noindent
  \insertfootnotemark\insertfootnotetext}

\let\beamerfn=\footnote
\renewcommand{\footnote}[1]{%
\let\oldfnsize=\footnotesize%
\let\footnotesize=\tiny%
\beamerfn<\thebeamerpauses->{#1}%
\let\footnotesize=\oldfnsize}


\date{\today}

\usepackage{eurosym}  
 
\renewcommand{\centerline}[1]{\hfill#1\hfill\hfill\mbox{}}


\title{Book publications
and sustainability}
% \institute{FU Berlin}
\author[Nordhoff]{Sebastian Nordhoff}
\date{Center for the Study of Language and Society Berne, 2019-05-29}


\begin{document}
\lspbeamertitle

\section{Introduction}

\frame{
\frametitle{Introduction}
%   \includegraphics[height=.2\textheight]{./path/to/graphicsfile}
  \begin{itemize}
    \item  PhD 2009 A grammar of Upcountry Sri Lanka Malay
    \item 2009-2012 Glottolog.org at the Max Planck Institute for Evolutionary Anthropology
    \item since 2014 coordinator for Language Science Press
  \end{itemize}
}

\frame{
\frametitle{Language Science Press}
%   \includegraphics[height=.2\textheight]{./path/to/graphicsfile}
  \begin{itemize}
    \item  scholar-owned, community-based publisher 
    \item Open Access
    \item free to read 
    \item free to publish 
    \item 25 series 
    \item 30 books a year (monographs and edited volumes)
  \end{itemize}
}

\frame{
\frametitle{Production of a book: traditional model}
%   \includegraphics[height=.2\textheight]{./path/to/graphicsfile}
  \begin{itemize}
    \item inquiry
	\item book proposal
	\item proposal approval
	\item contract
	\item first submission
	\item review
	\item first decision
	\item revision
	\item final decision
	\item copyediting
	\item first proofs
	\item typesetting
	\item final proofs
  \end{itemize}
}


\section{Technical aspects}
\frame{
\frametitle{Components}
%   \includegraphics[height=.2\textheight]{./path/to/graphicsfile}
  \begin{itemize}
    \item  Data (recordings, surveys, samples)
    \item  Code (scripts in R, python, praat)
    \item Prose (The text surrounding your data)
    \item Bibliography
    \item Proportions of these may vary
    \item Do keep them separate and well-organized
    \begin{itemize}
      \item do use a references manager like Zotero
    \end{itemize}
  \end{itemize}
}

\frame{
\frametitle{Versioning}
%   \includegraphics[height=.2\textheight]{./path/to/graphicsfile}
  \begin{itemize}
    \item Versioning is crucial for large projects 
    \item Back-up 
    \item Roll-back
    \item Fork
    \item Collaborate 
    \item Merge
    \item Automated tests
    \item Get feedback
  \end{itemize}
}

\frame{
\frametitle{Versioning: history}
  \includegraphics[height=\textheight]{juskancommits.png}
}


\frame{
\frametitle{Versioning: diffs}
  \includegraphics[height=\textheight]{juskansplit.png}
}


\frame{
\frametitle{Versioning: internal}
  \includegraphics[height=\textheight]{bodowinter.png}
  }

\frame{
\frametitle{Versioning and neat project structure}
  \includegraphics[height=\textheight]{pngR.png}
}  



\frame{
\frametitle{Versioning and neat project structure}
  \includegraphics[height=\textheight]{folderstructure.png}
}  

\frame{
\frametitle{Versioning and neat project structure}
  \includegraphics[height=\textheight]{NitzkecsvLO.png}
}  


\frame{
\frametitle{Versioning: releases}
  \includegraphics[height=\textheight]{unicodecookbook.png}
}  

\frame{
\frametitle{Versioning: releases}
  \includegraphics[height=\textheight]{releases.png}
}  


\frame{
\frametitle{Repositories}
%   \includegraphics[height=.2\textheight]{./path/to/graphicsfile}
  \begin{itemize}
    \item Repositories collect documents and research data 
      \begin{itemize}
        \item discpline-specific or general purpose
        \item preprint or postprint 
        \item text or data 
      \end{itemize}
  \end{itemize}
}

\frame{
   \includegraphics[height=\textheight]{figshare.png}
}

% \frame{
%    \includegraphics[height=\textheight]{dryad.png}
% }

\frame{
   \includegraphics[height=\textheight]{trolling.png}
}

\frame{
   \includegraphics[height=\textheight]{lingbuzz.png}
}

\frame{
   \includegraphics[height=\textheight]{semanticsarchive.png}
}

\frame{
   \includegraphics[height=\textheight]{roa.png}
}

\frame{
   \includegraphics[height=\textheight]{zenodo.png}
}

\frame{
   \includegraphics[height=\textheight]{github.png}
}


\frame{
\frametitle{Academia and Researchgate}
  \begin{itemize}
    \item  Academia.edu and researchgate are NOT repositories
    \item both make money from restricting access
    \end{itemize}
}                    

\frame{
\frametitle{A note on scopping}
  \begin{itemize}
    \item  Scooping means that someone ``steals'' your research while you are still finalising it
    \item that fear is by and large unwarranted 
    \item you can count yourself happy if anybody IS actually interested in your data!
    \item most research actually struggles a lot more with a lack of interest to outsiders rather than scooping 
    \item preprint servers allow you to register your data and establish primacy
  \end{itemize}
}
            

\frame{
\frametitle{Recommendations for technical sustainability}
  \begin{itemize}
    \item conceptually separate your data, code, and prose text   
    \item make sure you can replicate all steps of your analysis
    \item ideally, specify automated pipelines for your analysis
    \item do not hardwire data and code
    \item use a citation manager 
    \item use a versioning tool 
    \begin{itemize}
      \item local or university or cloud
    \end{itemize}
    \item use a preprint server
    \item backup\pause
    \item backup\pause
    \item backup
    \end{itemize}
}


            
\section{Legal aspects}

\frame{
\frametitle{Legal sustainability}
%   \includegraphics[height=.2\textheight]{./path/to/graphicsfile}
}

\frame{
\frametitle{Intellectual Property Rights}
%   \includegraphics[height=.2\textheight]{./path/to/graphicsfile}
  \begin{itemize}
    \item  ``Copyright'' in the Anglo-Saxon countries
    \item  ``Urheberrecht'' in continental Europe 
    \item These are different!
    \item Once you create somehting, you can decide who can use it and how
    \item Choose wisely and be explicit!
  \end{itemize}
}

\frame{
\frametitle{License}
%   \includegraphics[height=.2\textheight]{./path/to/graphicsfile}
  \begin{itemize}
    \item  Usage rights can be differentiated as follows
    \begin{itemize}
      \item geographical restriction (only for Europe)
      \item type restrictions (only for print) 
      \item exclusive (no-one else has the right) vs. non-exclusive (other people may also get the rights)
      \item copyright transfer agreement
      \item total buy-out
    \end{itemize}
  \end{itemize}
}

\frame{
\includegraphics[height=\textheight]{copyrighttransfer.pdf}
}

\frame{
\frametitle{Contracts}
%   \includegraphics[height=.2\textheight]{./path/to/graphicsfile}
  \begin{itemize}
    \item  Publisher contracts restrict everybody, including yourself 
    \item once you sign away your copyright, you have no longer the right to use your own material 
    \item to reuse tables or graphics in subsequent works of yours, you must first ask the new rights holder for permission 
    \item chasing rights is incredibly annoying
    \item publishers will put your content behind a paywall, meaning that it can actually be more difficult to access once it is officially published than before
    \item publisher vary as to whether and when they allow books to be hosted in repositories  
  \end{itemize}
    }

\frame{
\includegraphics[height=\textheight]{brillnordhoff.png}
 }

\frame{ 
\frametitle{Recommendations for legal sustainability}
  \begin{itemize}
    \item Publish your works under a Creative Commons license, preferably CC-BY 
    \begin{itemize}
      \item see the Berlin|Budapest|Bethesda declarations on Open Access 
    \end{itemize}
    \item This license clearly states who the copyright holder is and how the content can be reused (typically requiring only the attribution of the original author)
    \item do not sign away your copyright \pause
    \item do not sign away your copyright \pause 
    \item do not sign away your copyright
  \end{itemize}
}
                
%                 The editor hereby assigns to the Publisher the full copyright in the Work [...]. Consequently, the Publisher shall have the exclusive right \textbf{throughout the world} to publish and sell th Work in \textbf{all languages}, in whole or in part, including, \textbf{without limitation}, \textbf{any} abridgement and substantial part thereof, in book form and in \textbf{any other form} including, without limitation, mechanical, digital electronic, and visual reproduction, electronic storage and retrieval systems, including internet and intranet delivery and \textbf{all other forms of electronic publication now known or hereinafter invented}. 
                
\section{Financial aspects}
\frame{
\frametitle{Financial aspects of sustainability}
  \includegraphics[height=\textheight]{calculation.jpg}
 {\tiny CC-BY-ND Dennis Skley \url{https://www.flickr.com/photos/dskley/14895278831}}
 }
 
\frame{
\frametitle{Print run}
%   \includegraphics[height=.2\textheight]{./path/to/graphicsfile}
  \begin{itemize}
    \item  How many books does a publisher print? 
  \end{itemize}
} 

\frame{
\includegraphics[height=\textheight]{printingcosts.png}
}

\frame{
\frametitle{Print run}
%   \includegraphics[height=.2\textheight]{./path/to/graphicsfile}
  \begin{itemize}
    \item  How many books does a publisher sell?
    \item How much does the author typicall earn?
  \end{itemize}
} 

\frame{
  \includegraphics[height=\textheight]{royalty.pdf}
}        

\frame{
\frametitle{Business models: reader pays}
\includegraphics[height=.2\textheight]{brillnordhoff.png}
  \begin{itemize}
    \item  item-based
    \item  bundle 
    \item  patron driven acquisition
  \end{itemize}
}

\frame{
\frametitle{Services provided by publishers}
%   \includegraphics[height=.2\textheight]{./path/to/graphicsfile}
  \begin{itemize}
    \item review facilitation (or not)
    \item  language polishing (or not)
    \item  copy-editing (or not)
    \item  typesetting (or not)
    \item  indexing (or not)
  \end{itemize}
}

\frame{
\frametitle{Business models: author pays}
%   \includegraphics[height=.2\textheight]{./path/to/graphicsfile}
  \begin{itemize}
    \item  Article processing charges (APCs)
    \item  Book processing charges (BPCs)
    \item Book chapter processing charges (BCPCs)
    \item What is the cost of publishing an article? \pause
    \item What is the cost of publishing a book? \pause
    \item \textit{Scielo} (Brazil) charges 90 USD for one article
    \item \textit{Nature Communications} charges 5200 USD for one article
    \item this explains the profit margins of 30-40\% of the large publishers (Springer, Wiley, Elsevier)
  \end{itemize}
}

\frame{
\frametitle{Business model: discipline pays}
%   \includegraphics[height=.2\textheight]{./path/to/graphicsfile}
  \begin{itemize}
    \item institutional funding 
    \begin{itemize}
      \item Uni Bern: \url{https://bop.unibe.ch/linguistik-online}
      \item Uni Hawaii: Journal of Language Documentation and Conservation
      \item Uni Dartmouth: Linguistic Discovery 
      \item MIT/LSA: Semantics and Pragmatics
    \end{itemize}
    \item no fees for readers, no fees for authors
  \end{itemize}
}

\frame{
\frametitle{Business model: decentralized funding}
%   \includegraphics[height=.2\textheight]{./path/to/graphicsfile}
  \begin{itemize}
    \item  Open Library of Humanities (300+ supporting institutions)
    \begin{itemize}
      \item Glossa 
      \item Journal of Portuguese Linguistics 
      \item Laboratory Phonology
      \item Italian Journal of Linguistics 
    \end{itemize}
    \item Language Science Press (100+ supporters)
    \item Open Library Politikwissenschaft (40+ supporting institutions)
  \end{itemize}
}

\frame{
\frametitle{More figures}
\includegraphics[height=.2\textheight]{cookbook.png}
\includegraphics[height=.2\textheight]{businessmodel.png}
}

\frame{
\frametitle{Financial aspects: recommendations}
%   \includegraphics[height=.2\textheight]{./path/to/graphicsfile}
  \begin{itemize}
    \item go for discpline-pays  
    \item else go for author pays if fund are available 
    \item else go for reader pays and deposit preprints/postprints in repositories. 
  \end{itemize}
}

\section{Sociological aspects}
\frame{
\frametitle{Sociological aspects}
%   \includegraphics[height=.2\textheight]{./path/to/graphicsfile}
}

\frame{
\frametitle{Functions of a publisher}
%   \includegraphics[height=.2\textheight]{./path/to/graphicsfile}
  \begin{enumerate}
    \item  registration \only<2->{preprint server} 
    \item   certification \only<2->{???}
    \item   dissemination  \only<2->{repository, social media, print on demand}
    \item   archiving    \only<2->{Zenodo}
  \end{enumerate}
} 

\frame{
\includegraphics[height=.2\textheight]{facebook.png}
}

\frame{
\includegraphics[height=.2\textheight]{cumulativeall10000.png}
}

\frame{
\frametitle{Certification of scientific quality is outsourced to commercial entities}
%   \includegraphics[height=.2\textheight]{./path/to/graphicsfile}
  \begin{itemize}
    \item Universities want to hire good researchers  
    \item The amount of research produced by all candidates combined is too much for the committee to read
    \item The committee use some proxy to judge quality: the brand of the publisher of the journal 
    \item good publishers, regular publishers, bad publishers, ...
    \item NB: quality selection is done by the editors, not by the publishers
    \item the process is more or less the same for all publishers
  \end{itemize}
}

\frame{
\frametitle{Brands}
%   \includegraphics[height=.2\textheight]{./path/to/graphicsfile}
  \begin{itemize}
    \item One a publisher is perceived as prestigious, a self-reinforcing loop starts: 
    \begin{itemize}
      \item researchers present better manuscripts to the publisher 
      \item the publisher has more material to choose from 
      \item the average quality of the publisher gets better 
      \item prestige goes up
      \item start over 
    \end{itemize}
    \item Crucially, there is not necessarily more work being done by prestigious publishers 
    \item But since the brand is owned by a company, they increase the prices beyond proportion
  \end{itemize}
}

\frame{
\frametitle{Proofs by a prestigious publisher}
  \includegraphics[height=\textheight]{dg.png}
  \includegraphics[height=.2\textheight]{semf.png}
  }

\frame{
\frametitle{Errors and Journal Impact Factor}
\includegraphics[height=\textheight]{jif.png}
}  
  
\frame{
\frametitle{It's the employability, stupid}
%   \includegraphics[height=.2\textheight]{./path/to/graphicsfile}
  \begin{itemize}
    \item Being employed as a professor between 40 and 65 years gives you 3\,000\,000 EUR income. 
    \item It is economically rational for each individual researcher to spend 5\,000+ EUR for an article in Nature if this improves your chances to become a professor
    \item It is economically rational for Nature to charge as much as they can since they are a public company
    \item But this sucks money out of research, which could better be used to improve teaching, fund postdocs or develop vaccines. 
    \item Two solutions:
    \begin{itemize}
      \item decouple research evaluation from the place of publication 
      \item move the brands from the for-profit-sphere to the non-profit-sphere 
    \end{itemize}
  \end{itemize}
}

\frame{
\frametitle{Recommendations for financial aspects}
%   \includegraphics[height=.2\textheight]{./path/to/graphicsfile}
  \begin{itemize}
    \item Choose a scholar-led community-owned publisher 
    \item             do not exacerbate libaries' burden
   \item          lobby agains stupid criteria of research assesement like the Journal Impact Factor
  \end{itemize}
}



            
\section{Political aspects}
\frame{
\frametitle{Political aspects}
%   \includegraphics[height=.2\textheight]{./path/to/graphicsfile}
}

\frame{
\frametitle{Political initiatives}
%   \includegraphics[height=.2\textheight]{./path/to/graphicsfile}
  \begin{itemize}
    \item  DEAL: Germany negotiates national OA licensing programmes with big publishers 
    \item  Plan S: Research funders will require 100\% Open Access from projects they fund 
    \begin{itemize}
    \item There are indications that both actions probably lead to an increase in authory-pays publishing fees
    \end{itemize}
    \item Sci-Hub and LibGen allow you to access paywalled content and are probably more successful an instrument than all political initiatives 
  \end{itemize}
}

\frame{
\frametitle{Pollitical miscellanea}
%   \includegraphics[height=.2\textheight]{./path/to/graphicsfile}
  \begin{itemize}
    \item  Swiss National Fund pays generous processing charges 
    \item Austrian FWF is more on the repository side 
    \item The San Francisco Declaration on Research Assessment (DORA) calls for a move away from journals/publishers as proxies for research evaluation. 
  \end{itemize}
}

\frame{
\frametitle{Choice of publisher according to JS Caux}
%   \includegraphics[height=.2\textheight]{./path/to/graphicsfile}
  \begin{itemize}
    \item Gold: Content is free for readers  
    \item Platinum: as above, and free for authors 
    \item Palladium: As above, and all financial statements are disclosed\\\pause
    \item Iron: pay-to-read\pause
    \item Lead: editorial and financial aspects are not hermetically decoupled
  \end{itemize}
}
            
\frame{
\frametitle{Wrap-up: recommendations for sustainable book publications}
%   \includegraphics[height=.2\textheight]{./path/to/graphicsfile}
  \begin{itemize}
    \item separate data, code, and text 
    \item use a versioning system
    \item publish all your data in appropriate repositories
    \item use preprint servers
    \item use the CC-0 license for data and the CC-BY license for text
    \item do not sign away your copyright
    \item care for your discipline, go for scholar-led community-owned publishers
  \end{itemize}
}
    
%\setcounter{framenumber}{\thelastpagemainpart}
\end{document}
