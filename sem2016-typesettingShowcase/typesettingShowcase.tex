\documentclass[aspectratio=43,serif]{beamer}
\setbeamersize{text margin left=5pt, text margin right=5pt, sidebar width left=0pt,sidebar width right=0pt}
\usepackage{tikz}
\usetikzlibrary{positioning}
\usetikzlibrary{fit}
\usetikzlibrary{shapes.geometric}
\usetikzlibrary{calc}
\usetikzlibrary{patterns}
\usetikzlibrary{decorations.pathreplacing}
\usetikzlibrary{decorations.markings}
\usetikzlibrary{arrows.meta}
\usepackage{pgfplots}
\usepackage{graphicx}
\usepackage{dcolumn}
\newcolumntype{d}[1]{D{.}{.}{#1}}
\usepackage{booktabs}
\usepackage{mathspec}
\setallmainfonts(Digits,Latin,Greek)[
	Ligatures={TeX,Common},
	Path=./fonts/,
	PunctuationSpace=0,							
	%Numbers={OldStyle,Proportional},				% for tables use \addfontfeatures{Numbers={Monospaced,Lining}}
	Numbers={Proportional},	% normal numbers			% for tables use \addfontfeatures{Numbers={Monospaced,Lining}}
	BoldFont = LinLibertine_RZ_B.otf ,				% semi-bold
	ItalicFont = LinLibertine_RI_B.otf ,			
	BoldItalicFont = LinLibertine_RZI_B.otf 		% semi-bold
	]{LinLibertine_R_B.otf}	
\usepackage{langsci/langsci-tbls}
\input{langsci/langsci-colors.def}
\begin{document}

\frame{{\tiny\sffamily (from Berghäll 2015)}
\begin{columns}[c]
\begin{column}{62mm}
 \includegraphics[width=\textwidth]{im/BerghallMeanDegrees.jpeg}
\end{column}
\hfill%
\begin{column}{62mm}
 \resizebox{\textwidth}{!}{\pgfdeclarelayer{bg}
\pgfdeclarelayer{bg2}
\pgfsetlayers{bg2,bg,main}
\resizebox{\textwidth}{!}{
\begin{tikzpicture}[baseline]


\node	(moro)				at	(0,0)									{Moro};
\node	(meriman)			[below right=9mm and 21mm of moro]			{Meriman};
\node	(sapara)			[below right=6mm and 12mm of meriman]		{Sapara};
\node	(muaka)				[below=20mm of meriman]						{Muaka};
\node	(amiten)			[below left=20mm and 13mm of moro]			{Amiten};	
\node	(aketa)				[below=45mm of moro]		{Aketa};
\node	(saramun)			[right=46.75mm of aketa]						{Saramun};
\node	(yeipamir)			[below right=9mm and 18mm of aketa]			{Yeipamir};
\node	(papur)				[below right=24mm and 0mm of aketa]			{Papur};
\node	(tarigapa)			[right=30mm of papur]						{Tarigapa};
\node	(ulingan)		  	[right=40mm of muaka]						{Ulingan};
\node	(meiwok)			[right=30mm of ulingan]						{Meiwok};
\node	(sikor)				[right=39mm of saramun]		{Sikor};

\node	(susure)			[below=15mm of amiten]						{(Susure)};
\node	(wakoruma)			[below left=8mm and 17mm of moro]			{(Wakoruma)};


% Categories
\begin{pgfonlayer}{bg}
\node (kat1) [fill=gray!15, draw, dashed, rounded corners=15, fit=(moro) (meriman) (sapara) (muaka)] {};
\node (kat2) [fill=gray!15, draw, dashed, rounded corners=15, fit= (ulingan) (sikor) (meiwok)] {};
\node (kat3) [fill=gray!15, draw, dashed, rounded corners=15, fit= (papur) (yeipamir) (tarigapa)] {}; 
\end{pgfonlayer}
\begin{pgfonlayer}{bg2}
\node (kat4) [fill=gray!5,draw, dotted, rounded corners=15, fit = (kat1) (amiten) (aketa) (saramun), inner sep=10pt]{}; 
\end{pgfonlayer}

\footnotesize{
% Moro Relations
\path	(moro)	edge node [sloped, above] {0.04}							(meriman);
\path (moro)	edge [bend left] node [sloped, above] {0.06}	(sapara);
\path (moro)	edge [bend right=10] node [sloped, above] {0.03} (muaka);
\path (moro) 	edge node [sloped, below] {0.07} (aketa);
\path (moro)	edge [bend right] node [sloped, above] {0.09} (amiten); 

% Meriman Relations
\path (meriman) 	edge [sloped, above] node {0.04} (sapara);
\path (meriman) 	edge node [sloped, above] {0.04} (aketa);
\path (meriman) 	edge node [sloped, above, pos=0.333] {0.07} (muaka);

% Sapara Relations
\path (sapara) 	edge [bend right=20] node [sloped, above right] {0.05} (aketa);
\path (sapara) 	edge node [sloped, above] {0.07} (ulingan);
\path (sapara) 	edge node [sloped, below left] {0.05} (saramun);
\path (sapara) 	edge node [sloped, above left] {0.08} (muaka);

% Amiten Relations
\path (amiten)	edge node [sloped, below] {0.07}	(muaka);
\path (amiten)	edge [bend right] node [sloped,below left] {0.09}	(aketa);

% Muaka Relations
\path (muaka)	edge node [sloped, above right] {0.11}	(ulingan);
\path (muaka)	edge node [sloped, below] {0.08}	(saramun);
\path (muaka)	edge node [sloped, above] {0.11}	(tarigapa);
\path (muaka)	edge [bend left=10] node [sloped, below, pos=0.575] {0.10}	(yeipamir);
\path (muaka)	edge node [sloped, below] {0.11}	(aketa);

% Ulingan Relations
\path (ulingan) edge node [sloped, below] {0.12}	(saramun);
\path (ulingan) edge node [sloped, above] {0.15}	(tarigapa);
\path (ulingan) edge node [sloped, above] {0.02}	(sikor);
\path (ulingan) edge node [sloped, above] {0.03}	(meiwok);

% Aketa Relations
\path (aketa) edge node [sloped, below, pos=0.333] {0.08}	(papur);
\path (aketa) edge node [sloped, above] {0.08}	(yeipamir);
\path (aketa) edge node [sloped, below] {0.11}	(saramun);

% Saramun Relations
\path (saramun) edge node [sloped, below] {0.09}	(tarigapa);
\path (saramun) edge [bend left=35] node [above, sloped] {0.14}	(meiwok);
\path (saramun) edge node [sloped, above] {0.13}	(sikor);

% Sikor Relations
\path (sikor) edge node [sloped, above] {0.02}	(meiwok);
\path (sikor) edge node [sloped, below right] {0.14}	(tarigapa);

% Meiwok Relations
\path (meiwok) edge [bend left] node [sloped, below right] {0.17}	(tarigapa);

% Yeipamir Relations
\path (yeipamir) edge node [sloped, above] {0.03} (papur);
\path (yeipamir) edge node [sloped, below] {0.08} (tarigapa);

% Papur Relations
\path (papur) edge node [below] {0.11} 	(tarigapa);

% Tarigapa Relations
% Will be empty for now.
}
\end{tikzpicture}
}}
\end{column}%
\end{columns}}

\begin{frame}{\tiny\sffamily (from Fantinuoli et al (ed) 2015)}
\begin{columns}[T] % align columns
\begin{column}{60mm}
\centering\includegraphics[width=.5\textwidth]{im/Fan-4-1.png}\\\vspace{2em}\includegraphics[width=.5\textwidth]{im/Fan-4-3.png}
\end{column}%
\hfill%
\begin{column}{60mm}
\centering\resizebox{.45\textwidth}{!}{	\begin{tikzpicture}
	
	\node at (0,0) (EN) {English (EN)};
	\node [below=2cm of EN] (ES) {Spanish (ES)};
	\node [above=2cm of EN] (EL) {Greek (EL)};
	
	\node [above left=1cm and 1cm of EN, color=gray] (TL6) {TL6};
	\node [above right=1cm and 1cm of EN, color=gray] (TL3) {TL3};
	\node [below left=1cm and 1cm of EN, color=gray] (TL5) {TL5};
	\node [below right=1cm and 1cm of EN, color=gray] (TL4) {TL4};
	
	\draw [ultra thick, -{Triangle[]}] (node cs:name=EN, anchor=north) -- (node cs:name=EL, anchor=south);
	\draw [ultra thick, -{Triangle[]}] (EN.south) -- (ES.north);
	
	\draw [thick, gray, dotted, -{Triangle[]}] (EN.south west) -- (TL5.north east);
	\draw [thick, gray, dotted, -{Triangle[]}] (EN.north west) -- (TL6.south east);
	\draw [thick, gray, dotted, -{Triangle[]}] (EN.north east) -- (TL3.south west);
	\draw [thick, gray, dotted, -{Triangle[]}] (EN.south east) -- (TL4.north west);
	
	\end{tikzpicture}}\\\vspace{2em}\resizebox{.55\textwidth}{!}{%4-3-Corpus-Annotation-Scheme.tex
\begin{tikzpicture}
[every rectangle node/.style={inner sep=6pt, rounded corners=5, draw}]


\node at (0,0) [rectangle] (appraisal) {Appraisal};
\node [below=5mm of appraisal, rectangle] (appreciation) {Appreciation};
\node [right=20mm of appreciation.east, rectangle] (judgement) {Jugdement};
\node [left=20mm of appreciation.west, rectangle] (affect) {Affect};

\draw [thick] (node cs:name=appraisal, anchor=south) -| (node cs:name=appreciation);
\draw [thick] (node cs:name=affect, anchor=north) |- +(0,.275) -| (node cs:name=judgement);


\node [below right=10mm and 0mm of affect.west, rectangle, text width=40mm, align=center] (features) {Features for each type}; 
\node [below right=4mm and 10mm of features.south, rectangle, text width=40mm, align=center] (implicit) {implicit/explicit};
\node [below=4mm of implicit, rectangle, text width=40mm, align=center] (irony) {irony};
\node [below=4mm of irony, rectangle, text width=40mm, align=center] (indirect) {indirect/direct};
\node [below=4mm of indirect, rectangle, text width=40mm, align=center] (polarity) {polarity};
\node [below=4mm of polarity, rectangle, text width=40mm, align=center] (strength) {strength};

\draw [thick] (node cs:name=features, anchor=south) |- (node cs:name=implicit);
\draw [thick] (node cs:name=features, anchor=south) |- (node cs:name=irony);
\draw [thick] (node cs:name=features, anchor=south) |- (node cs:name=indirect);
\draw [thick] (node cs:name=features, anchor=south) |- (node cs:name=polarity);
\draw [thick] (node cs:name=features, anchor=south) |- (node cs:name=strength);

\end{tikzpicture}}
\end{column}%
\end{columns}
\end{frame}


\begin{frame}{\tiny\sffamily (from Payne et al (ed) 2016)}
\begin{columns}[c] % align columns
\begin{column}{60mm}
\includegraphics[width=\textwidth]{im/moodie-fig-2.png}
\end{column}%
\hfill%
\begin{column}{40mm}
\resizebox{\linewidth}{!}{\begin{tikzpicture}[trim axis left, trim axis right]
		\begin{axis}[ybar=0pt, ylabel={Number of occurences in lexicon}, xlabel={Numbering pattern},symbolic x coords={single, pairs or groups}, xtick=data, xtick align=center, bar width=2em, ymin=0, nodes near coords, nodes near coords align=vertical, xlabel near ticks, ymajorgrids, area legend, enlarge y limits={abs=2, upper}, enlarge x limits=0.5, legend entries={plurative,replacement,singuative}, legend style={cells={anchor=west}, legend pos=south east}]
		\addplot[pattern=dots] coordinates {(single,10)(pairs or groups,10)};
		\addplot[pattern=crosshatch] coordinates {(single,4)(pairs or groups,7)};
		\addplot[pattern=horizontal lines] coordinates {(single,0)(pairs or groups,9)};
		\end{axis}
		\end{tikzpicture}}
\end{column}%
\end{columns}
\end{frame}

\frame{{\tiny\sffamily (from Drager 2015)}\fboxsep=0pt
\noindent%
\begin{minipage}[t]{0.48\linewidth}
\begin{table}[htbp]
\begin{center}
\resizebox{60mm}{!}{
\begin{tabular}{|l|r|r|r|}
   \hline
   feature & CR girl &  NCR girl & CR and NCR \\
   questions comparing & quote$-$dp &  quote$-$dp &  quote$-$dp \\
  \hline
  total number subjects & 23 & 19 & 42 \\
  total questions answered & 916 & 774 & 1690 \\
  total 1st token labeled as quote & 465 & 383 & 848 \\
  quote first on sheet & 278 & 231 &  509 \\
  1st token's context more likely & 110 & 73 &  183 \\
  1st and 2nd tokens' context matched & 271 & 242 & 513 \\
  1st token's context less likely & 84 & 68 & 152 \\
  1st token mean EucD & 1.5930 & 1.5400  & 1.5690 \\
  2nd token mean EucD & 1.6180 & 1.6720 & 1.6430 \\
  mean EucD diff. (Bark) & -0.02538  & -0.13280 &  -0.07388 \\
  1st token mean nuc F2 (Bark) & 11.25  & 11.19 & 11.23 \\
  2nd token mean nuc F2 (Bark) & 11.49 & 11.45 & 11.47 \\
  mean nuc F2 diff. (Bark) & -0.2379 & -0.2554  & -0.2458 \\
 	1st token mean duration ratio & 0.33900  & 0.32670 & 0.33350 \\
 	2nd token mean duration ratio & 0.35020  & 0.34520  & 0.34790 \\
  mean duration ratio diff. & -0.01120  & -0.01844 &  -0.01447 \\
  1st token $[$k$]$ present, 2nd token $[$k$]$ absent & 93  & 84  & 177\\
  1st token $[$k$]$ absent, 2nd token $[$k$]$ present & 74  & 65  & 139 \\
  $[$k$]$ present for both tokens & 118 & 83 & 201 \\
  $[$k$]$ absent for both tokens & 180  & 151 & 331 \\
    \hline
    
  
  \end{tabular}
}
% \caption{Charcteristics of quote$-$dp questions in Experiment 1 where the first token was identified as the quotative, by whether the participant was in a CR or a NCR group.}\label{tab:percExp1responsesqd2}
\end{center}
\end{table}\end{minipage}%
\hfill%
\begin{minipage}[t]{0.48\linewidth}
\begin{table}[htbp]
\begin{center}
\resizebox{60mm}{!}{
\begin{tabular}{ld{5}d{5}d{5}}
   \toprule\toprule
    & \multicolumn{3}{c}{features} \\\cmidrule(lr){2-4} 
   questions comparing quote$-$dp& \multicolumn{1}{c}{CR girl} &  \multicolumn{1}{c}{NCR girl} & \multicolumn{1}{c}{CR and NCR} \\
  \midrule
  total number subjects & 23 & 19 & 42 \\
  total questions answered & 916 & 774 & 1690 \\
  total 1st token labeled as quote & 465 & 383 & 848 \\
  quote first on sheet & 278 & 231 &  509 \\
  1st token's context more likely & 110 & 73 &  183 \\
  1st and 2nd tokens' context matched & 271 & 242 & 513 \\
  1st token's context less likely & 84 & 68 & 152 \\
  1st token mean EucD & 1.5930 & 1.5400  & 1.5690 \\
  2nd token mean EucD & 1.6180 & 1.6720 & 1.6430 \\
  mean EucD diff. (Bark) & -0.02538  & -0.13280 &  -0.07388 \\
  1st token mean nuc F2 (Bark) & 11.25  & 11.19 & 11.23 \\
  2nd token mean nuc F2 (Bark) & 11.49 & 11.45 & 11.47 \\
  mean nuc F2 diff. (Bark) & -0.2379 & -0.2554  & -0.2458 \\
 	1st token mean duration ratio & 0.33900  & 0.32670 & 0.33350 \\
 	2nd token mean duration ratio & 0.35020  & 0.34520  & 0.34790 \\
  mean duration ratio diff. & -0.01120  & -0.01844 &  -0.01447 \\
  1st token $[$k$]$ present, 2nd token $[$k$]$ absent & 93  & 84  & 177\\
  1st token $[$k$]$ absent, 2nd token $[$k$]$ present & 74  & 65  & 139 \\
  $[$k$]$ present for both tokens & 118 & 83 & 201 \\
  $[$k$]$ absent for both tokens & 180  & 151 & 331 \\
    \bottomrule
\bottomrule    
  
  \end{tabular}
}
% \caption{Characteristics of quote$-$dp questions in Experiment 1 where the first token was identified as the quotative, by whether the participant was in a CR or a NCR group.}\label{tab:percExp1responsesqd}
\end{center}
\end{table}\end{minipage}%
}

\begin{frame}{\tiny\sffamily (from Friesen 2016)}
\begin{columns}[c] % align columns
\begin{column}{60mm}
\includegraphics[width=\textwidth]{im/Moloko-Table5.pdf}
\end{column}%
\hfill%
\begin{column}{60mm}
\includegraphics[width=\textwidth]{im/Moloko-new.pdf}
\end{column}%
\end{columns}
\end{frame}

\begin{frame}{\tiny\sffamily (from Schäfer 2016)}\footnotesize
\tblsfr[lsYellow]{law}{Kompositionalität}{Die Bedeutung komplexer sprachlicher Ausdrücke ergibt sich aus der Be-
deutung ihrer Teile und der Art ihrer grammatischen Kombination. Diese
Eigenschaft von Sprache nennt man \emph{Kompositionalität}.}
\tblsfi{Weiterführende Literatur}{Eine sehr ausführliche Einführung in die artikulatorische Phonetik
ist Laver (1994). Einführende Darstellungen der deutschen Phonetik finden sich
z. B. in Rues u. a. (2009) und Wiese (2010).}
\end{frame}


\end{document}