\documentclass[handout]{beamer}

\usepackage{fontspec} 
% \usepackage{lsp-makros}
\useoutertheme{lsp}

\usepackage{lsptitle}

\def\two@digits#1{\ifnum#1<10 0\fi\number#1}
\def\mytoday{\two@digits{\number\day}.\two@digits{\number\month}.\number\year}


\usepackage{xspace,multicol}
\newcommand{\latex}{\LaTeX\xspace}
\usepackage{tikz}


\newcounter{lastpagemainpart}
\footnotesep0pt
\renewcommand{\footnoterule}{}
\usefootnotetemplate{
  \noindent
  \insertfootnotemark\insertfootnotetext}

\let\beamerfn=\footnote
\renewcommand{\footnote}[1]{%
\let\oldfnsize=\footnotesize%
\let\footnotesize=\tiny%
\beamerfn<\thebeamerpauses->{#1}%
\let\footnotesize=\oldfnsize}


\date{27.6.2018 HTWK Leipzig}

\usepackage{eurosym}  
 
\renewcommand{\centerline}[1]{\hfill#1\hfill\hfill\mbox{}}


\title{Language Science Press}
% \institute{FU Berlin}
\author[LangSci]{\mbox{Felix Kopecky, Language Science Press}
% \mbox{\tiny\url{https://github.com/langsci/lsp-presentations/blob/master/oat2017books/presentation.pdf}}
}



\begin{document}
\lspbeamertitle

\section{Hintergrund} 

\frame{
\frametitle{Hintergrund}
%   \includegraphics[height=.2\textheight]{./path/to/graphicsfile}
  \begin{itemize}
    \item linguistische Monographien und Sammelbände
    \item international, Champions League 
    \item 100 \% CC-BY
    \item 2014 als DFG-Projekt gestartet, ab 2019 eigenfinanziert
    \item keine Kosten für Autor*innen und Leser*innen
    \item Software, Daten, Rohdateien, Geschäftsmodell frei verfügbar
%     \item Platinum open access (kostenlos für AutorInnen und LeserInnen)
%     \item Aus fachlichen Gründen digitale Bücher nur als PDF, Print-on-demand Angebote für Soft- und Hardcover, je nach Reihe
  \end{itemize}
}

\frame{
\frametitle{Zahlen}
    \begin{itemize}  
    \item  20 Reihen,  263 Herausgeber*innen aus 44 Ländern
    \item  68 veröffentlichte Bücher, 350 Interessensbekundungen
%     \item  950 \textit{public supporters} + 305 ``anonyme UnterstützerInnen''
    \item Ab 2018: 30 Bücher/Jahr
    \item bis zu >20.000 Downloads pro Buch
    \item 100 Bibliotheken und Institutionen weltweit unterstützen mit 1000 EUR/Jahr
    \item Open-Access-Preis der Deutschen Gesellschaft für Sprachwissenschaft
  \end{itemize}
}

\frame{
\frametitle{Veröffentlichte Bücher} 
    \includegraphics[width=\textwidth]{img/catalog.png}  
}
%
%\frame{ 
%\frametitle{Was wir publizieren:} 	
%\includegraphics[width=\textwidth]{catalog.png} 
%}



%\frame{ 
%\frametitle{Was wir publizieren:}
%\begin{tabular}{ll}
%\includegraphics[width=.35\textwidth]{gutman.png}& 
%\parbox{.65\textwidth}{
%\fbox{
%\includegraphics[width=.6\textwidth]{nena.png}  
%}
%\begin{itemize}
% \item Bücher bis 750 Seiten
% \item teilweise Frucht mehrerer Jahrzehnte Arbeit
% \item Mix zwischen Automatisierung und Maßanfertigung
%\end{itemize}
%
% 
%\vspace*{6cm}  
%}
%\end{tabular}
%}

% \frame{ 
% 	\frametitle{Publikationsworkflow} 	
% 	\includegraphics[width=\textwidth]{img/workflow.png} 
% }

\section{Bucherstellung}
\frame{
	\frametitle{Fluider Publikationsworkflow}
%	\begin{columns}
%		\begin{column}{5cm} 
			\begin{center}
				\vspace*{-2mm}
				\small
				\fbox{Einreichung}
				
				$\downarrow$
				
				\fbox{(open) peer review \raisebox{-0.5mm}{\includegraphics[width=3mm]{img/pdf.jpg}}}
				
				$\downarrow$
				
				\fbox{Überarbeitung durch AutorIn}
				
				$\downarrow$
				
				\fbox{Konversion zu \latex (semi-automatisch)}
				
				$\downarrow$
				
				\fbox{Community proofreading \raisebox{-0.5mm}{\includegraphics[width=3mm]{img/pdf.jpg}}}
				
				$\downarrow$
				
				\fbox{Satz durch AutorIn und LangSci \raisebox{-0.5mm}{\includegraphics[height=3mm]{img/overleaf.png}}}
				
				$\downarrow$
				
				\fbox{Veröffentlichung \raisebox{-0.5mm}{\includegraphics[width=3mm]{img/pdf.jpg}~\includegraphics[width=3mm]{img/doi.png}}}
				
				$\downarrow$
				
				\fbox{Neuauflagen \raisebox{-0.5mm}{\includegraphics[width=3mm]{img/pdf.jpg}~\includegraphics[width=3mm]{img/doi.png}}}
				
			\end{center}
%		\end{column}
%		\begin{column}{4.2cm}
%			\begin{itemize}
%				\item Versionen des Dokuments sind als pdf in verschiedenen Stadien verfügbar
%				\item Kein Anspruch auf Gatekeeper-Funktion
%			\end{itemize}
%			\vspace*{2cm}
%		\end{column}
%	\end{columns}
}

\section{Aussichten}

\frame{
\frametitle{Zukunft}
\begin{itemize}
 \item Open Access $\to$ Open Science 
 \item Wachstum: organisch/disruptiv/gedeckelt? 
 \item Übertragbarkeit auf andere Disziplinen: OpenAire-Projekt ``Full disclosure''
 \item Professionalisierung der Konsortialmodelle
 \item Einspeisung von OA-Publikationen in Bibliothekskataloge?
\end{itemize}

}



% \frame{ 
% \frametitle{Was wir publizieren:}
% \begin{tabular}{ll}
% \includegraphics[width=.35\textwidth]{nerbonne.png}& 
% \parbox{.65\textwidth}{
% \fbox{
% \includegraphics[width=.6\textwidth]{picklmap.png}  
% }
%  
% \vspace*{53mm}  
% }
% \end{tabular}
% }

 


% \frame{ 
% \frametitle{Was wir publizieren:} 
% \begin{tabular}{ll}
% \includegraphics[width=.35\textwidth]{mueller.png}& 
% \parbox{.65\textwidth}{
% \fbox{
% \includegraphics[width=.6\textwidth]{lfg.png}  
% }
%  
% \vspace*{50mm}  
% }
% \end{tabular}
% }
%
%  
%\frame{
%\frametitle{Timeline}
%\begin{itemize}
% \item   2012 erstes Treffen
% \item   2013 DFG-Antrag
% \item   2014-2016 DFG-Projekt
% \item   Seit 2017 an der HU
%\end{itemize}
%}    
%  
%\frame{
%\frametitle{Prinzipien}
%\begin{itemize}
% \item Prinzip der Offenheit
% \item Prinzip der Community 
% \item Prinzip der Schlankheit
%\end{itemize}
%}  


%\section{Prinzipien}
%% \subsection{Prinzip der Offenheit}
%
%\frame{
%\frametitle{Prinzip der Offenheit}
%%   \includegraphics[height=.2\textheight]{./path/to/graphicsfile}
%  \begin{itemize}
%    \item  Nur FLOSS, nur CC-BY, transparente Kalkulationen
%  \end{itemize}
%  
%  \includegraphics[width=3cm]{omp.png} 
%  \includegraphics[width=3.5cm]{github.png} 
%  \includegraphics[width=4cm]{paperhive.png}
%  
%  \includegraphics[width=3cm]{overleaf.png}
%  \includegraphics[width=3cm]{ctan.png}
%  \includegraphics[width=3cm]{oapen.png}
%  \includegraphics[width=3cm]{doab.png}
%}
% 
%
%% \subsection{Prinzip der Community}
%\frame{
%\frametitle{Prinzip der Community}
%%   \includegraphics[height=.2\textheight]{./path/to/graphicsfile}
%  ~\includegraphics[width=.95\textwidth]{WORLDMAPDOTSdots.png}
%  \begin{itemize}
%    \item weltweit, autark, dezentral, bottom-up
%    \item own the brands (anders als LivingReviews, SSRN,  etc)
%    \item share the source
%    \begin{itemize}
%     \item  Templates, Quelldateien, Geschäftsprozesse, Kalkulationen
%    \end{itemize}
%  \end{itemize}
%}
%
%
%% \subsection{Prinzip der Schlankheit}    
%
%\frame{
%\frametitle{Prinzip der Schlankheit}
%%   \includegraphics[height=.2\textheight]{./path/to/graphicsfile}
%  \begin{itemize}
%    \item keine Legacy-Software
%    \item keine Lagerhaltung
%    \item kein Vertrieb
%    \item keine IT für Paywalls, Registrierung
%    \item kein Marketing 
%    \item keine Buchstände
%    \item keine komplizierten Autorenverträge 
%    \item keine Tantiemen\\$\to$ born digital
%  \end{itemize}
%%}
%
%\section{Prestige}
%\frame{
%\frametitle{Prestige}
%\begin{itemize}
%\item Motivationsgefüge der Autoren berücksichtigen
%\begin{itemize}
% \item Karma $\Longleftrightarrow$ Karrierechancen
%\end{itemize}
%\item Karrierechancen sind ein wesentlicher Faktor bei der Wahl eines Verlages 
%\item Open Access kann nur dann Bestand haben, wenn die Karrierechancen nicht darunter leiden
%\item Karrierechancen korrelieren mit Prestige der Veröffentlichungsorte
%\item $\to$ ein neuer Verlag muss sehr schnell sehr viel Prestige aufbauen
%\end{itemize}
%}
%
%\frame{
%\frametitle{Quellen von Prestige}
%\begin{enumerate}
% \item Prominente Unterstützer
% \item Menge an Unterstützern
% \item Qualität der Bücher
% \item Selektivität/Exklusivität
%\end{enumerate}
%}
%
%\frame{
%\frametitle{Prominente Unterstützer}
%
%    \parbox{.4\textwidth}{
%      \includegraphics[height=.8\textheight]{pics/steels-s.jpg}
%      
%      Luc Steels
%      }
%    \parbox{.4\textwidth}{
%      ~\includegraphics[height=.8\textheight]{pics/pinker-s.jpg}
%      
%      Steven Pinker
%    }
%}
%
%
%
%\frame{
%\frametitle{Menge an Unterstützern}
%\begin{itemize}
%\item    öffentlich einsehbare Supporterseite 
%\begin{itemize}
% \item \url{http://langsci-press.org/supporters}
%\end{itemize}
%\item    Mailingliste
%\item    ``Pledge to Publish'' vor Projektstart
%\item    kritische Masse war schon vor Projektstart erreicht
%\item    Start auf der Jahrestagung der Deutschen Gesellschaft für Sprachwissenschaft 
%mit dem ersten Buch 7 Tage nach Projektbeginn
%\end{itemize}
%}
%
%
%\frame{
%\frametitle{Formale Qualität}
%\begin{itemize}
%\item Einheitliches reihenübergreifendes Erscheinungsbild 
%\item \LaTeX
%\item Ligaturen 
%\parbox{.3\textwidth}{\includegraphics[width=.3\textwidth]{pics/affin.png}}
% \item recht strenges Stylesheet
%\end{itemize}
%}
%
%\frame{
%\frametitle{Formale Qualität}
%\begin{itemize}
%\item Chinesisch, Hebräisch, Arabisch
%\item diverse fachspezifische Erweiterungen
%\item Namensindex, Sprachindex, thematischer Index
%\item klickbare Querverweise
%\item Print-on-Demand
%\begin{itemize}
%  \item ISBNs
%  \item DOIs, auch auf Kapitelebene
% \item theoretisch noch Prestige durch hohen Preis im dreistelligen Bereich
% \item bei Language Science Press ca. 30 EUR/Buch
%\end{itemize}
%
%\end{itemize}
%}
%
%\frame{
%\frametitle{Inhaltliche Qualität}
%\begin{itemize}
%\item traditioneller Peer Review
%\begin{itemize}
% \item organisiert vom Reihenherausgeber zusammen mit Editorial Board
%\end{itemize}
%\item keine Dissertationen (Selektivität)
%\item proaktive Kommunikation von Annahme-/Ablehnungraten
%\end{itemize}
%}
%
%\frame{
%\frametitle{Innovation}
%\begin{itemize}
% \item GitHub 
% \item Overleaf 
% \item PaperHive
%\end{itemize}
%}
%\section{Finanzierung}
%
%\frame{
%\frametitle{Finanzierung}
%\begin{itemize}
% \item DFG-Projekt mit 50\%-Stelle für Betriebswirtin 
% \item Vision, Mission, Stakeholder-Analyse, Wertversprechen, Einnahmearten, Kostendeckungsbeiträge
% \item Overhead pro Kostenart
% \item Benötigt: ca. 100\,000 EUR p/a
% \item Finanzierungsmöglichkeiten \pause
% \begin{itemize}
%
% \item Consortia\pause
% \item Printmarge \pause 
% \item Autorengebühren\pause 
% \item Spenden\pause
% \item Mitgliedschaften  
% \end{itemize}
%\end{itemize}
%}
%
%\frame{
%\frametitle{Knowledge Unlatched}
%\begin{itemize}
% \item Verteilte Finanzierung: 100 Bibliotheken weltweit zu 1000 EUR/Jahr 
% \item Leistung: 30 Bücher/Jahr, peer-reviewed, CC-BY
% \item 3-jährige Perioden
% \item je nach Bibliothek sehr schnell oder sehr langwierig
%\end{itemize}
%\includegraphics[width=10cm]{pics/funding.png}
%}
%\section{Übertragbarkeit}
%
%% \frame{
%% \frametitle{Voraussetzungen} 
%%   \begin{itemize}
%%   \item Community-Building
%%   \item keine Gewinnerzielungsabsicht
%%   \item kein Anspruch auf Verwertungsrechtemonopol
%%   \item verteilte Finanzierung (konkret: Knowledge Unlatched)
%%   \item klares inhaltliches Profil
%%   \item Buchmenge überschaubar und vorhersagbar 
%%   \item Anerkennungskultur      
%%   \end{itemize} 
%% }
%
%\frame{
%\frametitle{Language Science Press zum Nachkochen}
%\begin{itemize}
% \item OpenAire-Projekt: \textit{Full disclosure: replicable strategies for book publications supplemented with empirical data}
% \begin{itemize}
% \item Geschäftsmodell zum Nachlesen als CC-BY 
% \item HowTo zum Nachbauen als CC-BY
% \item Geschäftszahlen (Downloads, Verkäufe) als CC-0
% \item Spreadsheet zum Nachrechnen
% \item Verfügbar 2018/06
% \end{itemize}
%\end{itemize}
%\includegraphics[width=2cm]{pics/openaire.png}
%}
%
%\section{Diskussion}
%
%\frame{
%\frametitle{Diskussion}
%\begin{itemize}
%  \item OA-``Heilsversprechen''
% \begin{itemize}
% \item Haben durch OA mehr Menschen Zugang? 
% \item Ist OA billiger? 
% \item Ist OA demokratischer? 
% \end{itemize}
% \item Was sind die Voraussetzungen innerhalb eines Fachbereichs für eine erfolgreiche OA-Transformation? 
% \begin{itemize}
%  \item Hochenergiephysik (SCOAP3)
%  \item Sprachwissenschaft (LingOA, Glossa, Language Science Press)
%  \item ???
% \end{itemize}
% \item Sammlungsauftrag von Bibliotheken vs. dezentrale Finanzierung von Publikationsplattformen
%\end{itemize}
%}

\end{document}
