\documentclass[handout]{beamer}

\usepackage{fontspec} 
% \usepackage{lsp-makros}
\useoutertheme{lsp}

\usepackage{lsptitle}

\def\two@digits#1{\ifnum#1<10 0\fi\number#1}
\def\mytoday{\two@digits{\number\day}.\two@digits{\number\month}.\number\year}


\usepackage{xspace,multicol}
\newcommand{\latex}{\LaTeX\xspace}
\usepackage{tikz}


\newcounter{lastpagemainpart}
\footnotesep0pt
\renewcommand{\footnoterule}{}
\usefootnotetemplate{
  \noindent
  \insertfootnotemark\insertfootnotetext}

\let\beamerfn=\footnote
\renewcommand{\footnote}[1]{%
\let\oldfnsize=\footnotesize%
\let\footnotesize=\tiny%
\beamerfn<\thebeamerpauses->{#1}%
\let\footnotesize=\oldfnsize}


\date{\today}

\usepackage{eurosym}  
\usepackage{multicol}
\renewcommand{\centerline}[1]{\hfill#1\hfill\hfill\mbox{}}


\title{LangSci Series Editors Meeting 2023}
\institute{2023-11-17, HU Berlin}
\author[LangSci]{LangSci}



\begin{document}
\lspbeamertitle

\section{Funding}
\frame{
\frametitle{Funding}
%   \includegraphics[height=.2\textheight]{./path/to/graphicsfile}
  \begin{itemize}
    \item  First funding round  2018--2020
    \begin{itemize}
      \item \textbf{105} institutions * \textbf{1000} EUR = 100k€
    \end{itemize}
    \item Second funding round 2021--2023
    \begin{itemize}
      \item \textbf{116} institutions * \textbf{1000} EUR = 116k€
    \end{itemize}
    \item Target third founding round 2024--2026
    \begin{itemize}
      \item \textbf{100} institutions * \textbf{1300} EUR = 130k€
    \end{itemize}
  \end{itemize}
}

\frame{
\frametitle{Countries}
  \begin{itemize}
    \item  Lion's share
    \begin{itemize}
      \item \textbf{Germany}
    \end{itemize}
    \item  well represented
    \begin{itemize}
      \item \textbf{Netherlands}, \textbf{Switzerland}, \textbf{Belgium}
    \end{itemize}
    \item OK
    \begin{itemize}
      \item  \textbf{Sweden}, \textbf{Norway}, \textbf{Finland}, \textbf{Austria}
    \end{itemize}
    \item can do better
    \begin{itemize}
      \item \textbf{UK}, \textbf{US}, \textbf{Australia}, \textbf{France}
    \end{itemize}
    \item  low representation
    \begin{itemize}
      \item \textbf{Spain}, \textbf{Eastern Europe}
    \end{itemize}
      \item no representation
      \begin{itemize}
        \item \textbf{Asia}, \textbf{Africa}, \textbf{Middle East}, \textbf{Latin America}, \textbf{Denmark}, \textbf{Ireland}, \textbf{New Zealand}, \textbf{Italy}, \textbf{Portugal}
      \end{itemize}
  \end{itemize}
}
\section{Series evolution}
\frame{
\frametitle{Series evolution}
\begin{itemize}
  \item currently 35 series
  \item two more series accepted
  \begin{itemize}
    \item \textit{Current Issues in Celtic Linguistics}
    \item \textit{Frame-Based Approaches to Semantics}
  \end{itemize}
  \item rejected
  \begin{itemize}
    \item series on translation practices (not linguistic enough)
  \end{itemize}
  \item in preparation
  \begin{itemize}
    \item Papuan linguistics
  \end{itemize}
  \item decommissioned
  \begin{itemize}
    \item \textit{Morphological Investigations}
    \item \textit{Topics at the Grammar-Discourse Interface}
    \item \textit{Studies in Diversity Linguistics}
  \end{itemize}
  \item dormant
  \begin{itemize}
    \item \textit{Computational Models of Language Evolution}
    \item \textit{Classics in Linguistics}
  \end{itemize}
\end{itemize}
}

\section{Books}

\frame{
\frametitle{List of forthcoming books 2024 (33)}
\begin{multicols}{3}
\textcolor{lsLightOrange}{Childs ALGAD}\\
\textcolor{lsLightOrange}{Jenks ALGAD}\\
\textcolor{lsMidWine}{Green CAL}\\
\textcolor{lsMidWine}{Essegbey CAL}\\
\textcolor{lsRed}{Serrelli CAM}\\
\textcolor{lsLightGreen}{PutnamPolinsky CIB}\\
\textcolor{lsNightBlue}{Payne COGL}\\
\textcolor{lsNightBlue}{Terhart COGL}\\
\textcolor{lsMidDarkBlue}{Dyck ELA}\\
\textcolor{lsMidBlue}{Gibson EOTMS}\\
\textcolor{lsMidBlue}{Winckel EOTMS}\\
\textcolor{black}{Herkel HPLS}\\
\textcolor{black}{Stockigt HPLS}\\
\textcolor{lsDarkWine}{Andreassen LV}\\
\textcolor{lsDarkWine}{Wagner LV}\\
\textcolor{lsMidDarkBlue}{Bech OGL}\\
\textcolor{lsDarkWine}{Favaro ORL}\\
\textcolor{lsDarkWine}{Rosemeyer ORL}\\
\textcolor{lsDarkWine}{Howe ORL}\\
\textcolor{lsMidGreen}{Zimmermann OSL}\\
\textcolor{lsLightBlue}{Giouli PMWE}\\
\textcolor{lsDarkOrange}{Miestamo RCG}\\
\textcolor{lsDarkOrange}{Bartens  SCL}\\
\textcolor{lsRichGreen}{Enke SIDL}\\
\textcolor{lsLightWine}{Ahn SILP}\\
\textcolor{lsLightWine}{Fuchs SILP}\\
\textcolor{lsYellow}{Brunner TBLS}\\
\textcolor{lsYellow}{Neacşu TBLS}\\
\textcolor{lsYellow}{Vaissiere TBLS}\\
\textcolor{lsDarkBlue}{Tallman TPD}\\
\textcolor{lsDarkBlue}{Bracks TPD}\\
\textcolor{lsDarkBlue}{Nikulin TPD}\\
\textcolor{lsDarkBlue}{Kuznetsova TPD}\\
\end{multicols}
}


\frame{
\frametitle{List of forthcoming books 2025 (14+)}
\begin{multicols}{3}
\textcolor{lsRed}{Auer AHL}\\
\textcolor{lsRed}{Haig CAM}\\
\textcolor{lsRed}{Lamberti CAM}\\
\textcolor{lsRed}{Stoynova CAM}\\
\textcolor{lsRed}{Schroeder CAM}\\
\textcolor{lsLightGreen}{Soesman CIB}\\
\textcolor{lsNightBlue}{Niinaga COGL}\\
\textcolor{lsNightBlue}{Zahrer COGL}\\
\textcolor{lsNightBlue}{Rohleder COGL}\\
\textcolor{lsMidBlue}{Krifka EOTMS}\\
\textcolor{lsMidBlue}{Kuelpmann EOTMS}\\
\textcolor{lsDarkOrange}{Sinnemäki RCG}\\
\textcolor{lsDarkOrange}{Faller RCG}\\
\textcolor{lsYellow}{Richter TBLS}\\
\end{multicols}
}

\frame{
\frametitle{Priorisation of submissions}
\begin{itemize}
  \item In the next funding period 2024--2026, we will produce 33 books a year
  \item We will have 31 productive series, with 2 more accepted series
  \item $\sim 1$ book/year per series
  \item How to resolve competition for scarce slots?
  \begin{itemize}
    \item  proposal: series with less books in a given year goes first
  \end{itemize}
\end{itemize}
}

\frame{
\frametitle{Book policies}
\begin{itemize}
  \item We have been opposed to festschrifts from the beginning, but somewhat lenient
  \item We have suffered, and we have learned from our mistakes
  \item Festschrifts are by and large a nightmare
  \begin{itemize}
    \item  Stefan and Felix can attest to that
  \end{itemize}
  \item We will not publish anything which resembles a festschrift from now on. Period.
\end{itemize}
}

\frame{
\frametitle{Quality of submissions}
\begin{itemize}
  \item In general, the technical quality of submissions has significantly improved
  \item People use our templates, and series provide guidance to authors about what is expected from them
  \begin{itemize}
    \item CAL has workshops at their annual meetings and some kind of LaTeX response team.
    \item please share the experiences of your series
  \end{itemize}
   \end{itemize}
   }


\frame{
\frametitle{Quality of submissions}
\begin{columns}
\column{5.8cm}
\begin{itemize}
  \item One submission with very shoddy papers
  \item We cannot assure publication with the resources we have at our disposal.
  \item Cavalier disregard of any and all guidelines, which we cannot compensate.
  \item Volume editors will have to rework the papers, but they do not have the resources either
  \item Book stalled and unlikely to be published any time soon, or even at all.
\end{itemize}
\column{4.5cm}
~~\includegraphics[width=.8\textwidth]{bookinalley.png}
\begin{itemize}
  \item Given the increased output and workload, we will have less time for ``rescue missions'' for manuscripts in distress.
\end{itemize}

\end{columns}
}

\section{Open Text Collections}
\frame{
\frametitle{Open Text Collections }
  \begin{itemize}
    \item  New project joining the \textit{Berlin-Brandenburg Academy of Sciences}, the \textit{Endangered Language Documentation Programme}, and \textit{Language Science Press}
    \item Get interlinearized texts in a structured format, treating them as research data
    \item Generate HTML pages, PDFs, but also NLP-compatible formats like JSON or CSV
    \item Print-on-demand books to be published via LangSci
    \item All costs covered until 2026, marginal costs after that
  \end{itemize}
}

\frame{
\frametitle{Room for open questions}
  \begin{itemize}
    \item
  \end{itemize}
}

\section{Open Science}
\frame{
In the context of a new series proposal, the advisory board raised the question of Open Science.
\frametitle{Open Science}
  \begin{itemize}
    \item       \url{https://github.com/langsci/guidelines/issues/2}
  \end{itemize}
} 

\frame{
\frametitle{Open Science }
  \begin{itemize}
    \item  LangSci is Open Access, we should probably Open Science as well
    \item We already have
    \begin{itemize}
      \item open source software
      \item source code of the books available
      \item bibliographies on Glottolog
      \item examples in IMTVault
      \item editable graphics
    \end{itemize}
    \item we kind of have
    \begin{itemize}
      \item a data/ folder on the GitHub repositories where tabular data can be stored
      \item this is currently not well supported, nor used a lot
  \end{itemize}
  \end{itemize}
}

\frame{
\frametitle{Issues in the field of Open Science}
  \begin{enumerate}
    \item  Replicability.
    \begin{itemize}
      \item Researchers should be able to redo the analysis (or parts thereof) and verify the results
      \item Full chain of scientific argumentation
    \end{itemize}
    \item Reuse
    \begin{itemize}
      \item Data once collected should be available for further/other/different research down the road
    \end{itemize}
  \end{enumerate}
  \begin{itemize}
    \item Discuss whether replicability or reuse is our main focus.
  \end{itemize}

}

\frame{
\frametitle{FAIR Principles}
The first step in (re)using data is to find them. Metadata and data should be easy to find for both humans and computers. Machine-readable metadata are essential for automatic discovery of datasets and services, so this is an essential component of the FAIRification process.
  \begin{itemize}
    \item In the domain of reuse, the FAIR principles are a common yardstick
    \begin{itemize}
      \item Findable
      \item Accessible
      \item Interoperable
      \item Reusable
    \end{itemize}

    \item
  \end{itemize}
}

\frame{
\frametitle{Findable}
  \begin{itemize}
    \item[F1.] (Meta)data are assigned a globally unique and persistent identifier
    \item[F2.] Data are described with rich metadata (defined by R1 below)
    \item[F3.] Metadata clearly and explicitly include the identifier of the data they describe
    \item[F4.] (Meta)data are registered or indexed in a searchable resource
  \end{itemize}
}

\frame{
\frametitle{Accessible}
Once the user finds the required data, they need to know how the data can be accessed, possibly including authentication and authorisation.
  \begin{itemize}
    \item[A1.] (Meta)data are retrievable by their identifier using a standardised communications protocol
    \begin{itemize}
      \item[A1.1] The protocol is open, free, and universally implementable
      \item[A1.2] The protocol allows for an authentication and authorisation procedure, where necessary
    \end{itemize}
    \item[A2.] Metadata are accessible, even when the data are no longer available
  \end{itemize}
}



\frame{
\frametitle{Interoperable}
The data usually need to be integrated with other data. In addition, the data need to interoperate with applications or workflows for analysis, storage, and processing.
  \begin{itemize}
    \item[I1.] (Meta)data use a formal, accessible, shared, and broadly applicable language for knowledge representation.
    \item[I2.] (Meta)data use vocabularies that follow FAIR principles
    \item[I3.] (Meta)data include qualified references to other (meta)data
  \end{itemize}
}



\frame{
\frametitle{Reusable}
The ultimate goal of FAIR is to optimise the reuse of data. To achieve this, metadata and data should be well-described so that they can be replicated and/or combined in different settings.
  \begin{itemize}
    \item[R1.] (Meta)data are richly described with a plurality of accurate and relevant attributes
    \begin{itemize}
    \item[R1.1.] (Meta)data are released with a clear and accessible data usage license
    \item[R1.2.] (Meta)data are associated with detailed provenance
    \item[R1.3.] (Meta)data meet domain-relevant community standards
    \end{itemize}
  \end{itemize}
}

\frame{
\frametitle{Additional considerations}
  \begin{itemize}
    \item  third party copyrights
    \item personality rights
    \begin{itemize}
      \item audio
      \item video
      \item children
    \end{itemize}

  \end{itemize}
}

\frame{
\frametitle{Take-home messages?}
\begin{columns}
\column{5cm}
\begin{itemize}
  \item What does this mean for your subfield?
  \item What existing strucures can we use?
  \begin{itemize}
    \item Zenodo
    \item OSF
    \item Trolling
    \item DELAMAN
    \item ...
  \end{itemize}
\end{itemize}
\column{5cm}
  \begin{itemize}
    \item What should be recommend to authors?
    \item What should we recommend to future series?
    \item What should we recommend to existing series?
  \end{itemize}
\end{columns}
}















%\setcounter{framenumber}{\thelastpagemainpart}
\end{document}
