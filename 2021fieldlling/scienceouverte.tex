\documentclass[]{beamer}

\usepackage{fontspec} 
% \usepackage{lsp-makros}
\useoutertheme{lsp}

\usepackage{lsptitle}

\def\two@digits#1{\ifnum#1<10 0\fi\number#1}
\def\mytoday{\two@digits{\number\day}.\two@digits{\number\month}.\number\year}


\usepackage{xspace,multicol}
\newcommand{\latex}{\LaTeX\xspace}
\usepackage{tikz}


\newcounter{lastpagemainpart}
\footnotesep0pt
\renewcommand{\footnoterule}{}
\usefootnotetemplate{
  \noindent
  \insertfootnotemark\insertfootnotetext}

\let\beamerfn=\footnote
\renewcommand{\footnote}[1]{%
\let\oldfnsize=\footnotesize%
\let\footnotesize=\tiny%
\beamerfn<\thebeamerpauses->{#1}%
\let\footnotesize=\oldfnsize}


\date{\today}

\usepackage{eurosym}  
 
\renewcommand{\centerline}[1]{\hfill#1\hfill\hfill\mbox{}}
\newcommand{\highlight}{\mdseries\color{red!80!black}}

\title{Technical, legal, and social aspects of Open Science in linguistics}
% \institute{FU Berlin}
\author[Nordhoff]{Sebastian Nordhoff}
\date{CNRS-Villejuif, 2021-09-09}


% Open Science is a concept which can be applied to all stages of the
% research workflow in linguistics: we have the gathering of data, the
% transformation and enrichment of data via scripts or human annotation,
% and the publication of research findings in books or articles. Data,
% scripts, and publications should all be made as open as possible in
% order for other researchers to
%engage with and build upon the research
% being undertaken.
% In this talk, Sebastian Nordhoff will show how these aspects of openness
% can be realised during the production of a linguistic book, using the
% open access publisher Language Science Press as an example. Topics
% discussed include file formats and criteria to choose among them,
% workflows, collaborative editing and versioning, linguistic data
% repositories, linguistic preprint servers, disambiguation of documents
% and researchers, long term preservation, discoverability, reusability,
% legal aspects, and the impact of reputation and prestige.
% These aspects will be exemplified via the workflow of a manuscript,
% showing how and where at each stage there are potentials for making each
% step more open, transparent, and collaborative.

\begin{document}
\lspbeamertitle

\section{Introduction}

% \frame{
% \frametitle{Outline}
% \tableofcontents
% }

\frame{
\frametitle{Sebastian Nordhoff}
%   \includegraphics[height=.2\textheight]{./path/to/graphicsfile}
  \begin{itemize}
    \item PhD 2009 \textit{A grammar of Upcountry Sri Lanka Malay}
    \item 3 edited volumes, about 30 research articles
    \item 2009-2012 Glottolog.org at the Max Planck Institute for Evolutionary Anthropology
    \item advocate for Open Source, Open Access, Open Data, Open Everything 
    \item since 2014 coordinator for Language Science Press
    \item involved in the publication of 150+ books since 2014
  \end{itemize}
}

\frame{
\frametitle{Language Science Press}
%   \includegraphics[height=.2\textheight]{./path/to/graphicsfile}
  \begin{itemize}
    \item  scholar-owned, community-based publisher 
    \item Open Access
    \item free to read 
    \item free to publish 
    \item 29 series
    \item 30 books a year (monographs and edited volumes)
  \end{itemize}
}

\frame{
\frametitle{Language Science Press}
\includegraphics[height=\textheight]{catalog.png}
}
% 
% \frame{
% \frametitle{Production of a book:\\ traditional model}
% %   \includegraphics[height=.2\textheight]{./path/to/graphicsfile}
%   \begin{enumerate}
%     \item inquiry
% 	\item book proposal
% 	\item proposal approval
% 	\item contract
% 	\item first submission
% 	\item review
% 	\item first decision
% 	\item revision
% 	\item final decision
% 	\item copyediting
% 	\item first proofs
% 	\item typesetting
% 	\item final proofs
%   \end{enumerate}
% }


\frame{
\frametitle{Goals of open science}
%   \includegraphics[height=.2\textheight]{./path/to/graphicsfile}
  \begin{itemize}
    \item replicable/reliable
    \item inclusive
    \item participative
    \item collaborative
    \item transparent
    \item reusable
  \end{itemize}
}

\frame{
\frametitle{Scientific workflow}
%   \includegraphics[height=.2\textheight]{./path/to/graphicsfile}
  \begin{enumerate}
    \item gathering of data
    \begin{itemize}
      \item eg recordings, questionnaire
    \end{itemize}
    \item transformation
    \begin{itemize}
      \item cut, crop, cleanse
    \end{itemize}
    \item enrichment
    \begin{itemize}
      \item link to other data, analyze
    \end{itemize}
    \item publication
    \begin{itemize}
      \item make available
    \end{itemize}
  \end{enumerate}
}

\frame{
\frametitle{Steps}
%   \includegraphics[height=.2\textheight]{./path/to/graphicsfile}
  \begin{enumerate}
    \item pre-submission
    \item submission
    \item  review
    \item revision
    \item  proofreading
    \item publication\\~
    \item reading
    \item iteration
    \end{enumerate}
    \begin{itemize}
    \item prestige
    \item long-term preservation
  \end{itemize}
  }

\frame{
\frametitle{Pre-submission: object types }
\begin{columns}
\column{7cm}
  \begin{itemize}
   \item Scientific research uses different kinds of entities
   \begin{itemize}
    \item data
    \item scripts/processes/methods
    \item texts
   \end{itemize}
   \item Do keep them separate
  \end{itemize}
  \column{3cm}
  \includegraphics[height=\textheight]{pngR.png}
\end{columns}
}

\frame{
\frametitle{Pre-submission: formats}
%   \includegraphics[height=.2\textheight]{./path/to/graphicsfile}
  \begin{itemize}
    \item For your work to be accessible to others, you should use fileformats which are
    \begin{itemize}
    \item  standardized
    \begin{itemize}
      \item a skilled and motivated person should be able to write an application to read and write the format
    \end{itemize}
    \item   open
    \begin{itemize}
      \item royalty-free
    \end{itemize}
    \item   suited for tasks
    \begin{itemize}
      \item eg, do not use jpg for texts
    \end{itemize}
    \end{itemize}
    \item \url{https://en.wikipedia.org/wiki/List_of_open_formats}
  \end{itemize}
}

\frame{
\frametitle{Sample table}
\begin{itemize}
  \item what an easy table looks like to humans
\end{itemize}

\includegraphics[height=.2\textheight]{sampletable_oo.png}

\begin{itemize}
  \item what an easy table looks like to computers
\only<2>{\textbf{pdf}:\\ \includegraphics[height=4cm]{sampletable_pdf.png}}
\only<3>{\textbf{LibreOffice XML}:\\\includegraphics[height=8cm]{sampletable_xml.png}}
\only<4>{\textbf{csv}:\\ \includegraphics[height=3cm]{sampletable_csv.png}}
\end{itemize}
}

\frame{
\frametitle{Dissemination}
%   \includegraphics[height=.2\textheight]{./path/to/graphicsfile}
  \begin{itemize}
    \item For your work to be useful for others, those ``others'' must
    \begin{itemize}
      \item know that it exists (\textbf{find} it)
      \item be able to \textbf{access} it
    \end{itemize}
     \item preprint servers
     \begin{itemize}
       \item lingbuzz
       \item semanticsarchive
       \item Rutgers Optimality Archive
     \end{itemize}
  \end{itemize}
}

 \frame{
\frametitle{Publisher selection}
%   \includegraphics[height=.2\textheight]{./path/to/graphicsfile}
\begin{itemize}
  \item Fair Open Access Principles (\url{https://www.fairopenaccess.org/the-fair-open-access-principles})
\end{itemize}
\begin{enumerate}
\item    The journal has a \textbf{transparent ownership structure}, and is \textbf{controlled by} and responsive to the \textbf{scholarly community}.
\item     Authors of articles in the journal \textbf{retain copyright}.
\item    All articles are published open access and an \textbf{explicit open access licence} is used.
\item    \textbf{Submission and publication is not conditional} in any way on \textbf{the payment of a fee} from the author or their employing institution, or on membership of an institution or society.
\item    Any \textbf{fees} paid on behalf of the journal to publishers \textbf{are low, transparent, and in proportion} to the work carried out.
\end{enumerate}
}

\frame{
\frametitle{Publisher selection}
%   \includegraphics[height=.2\textheight]{./path/to/graphicsfile}
  \begin{itemize}
    \item
            journals: \url{oaling.wordpress.com} has a list of diamond (no fee) OA journals
    \item   books: not so many diamond options, besides LangSci Press
  \end{itemize}
}

\frame{
\frametitle{Submission}
%   \includegraphics[height=.2\textheight]{./path/to/graphicsfile}
  \begin{itemize}
    \item How could a more open and more inclusive submission process look like?
    \begin{itemize}
      \item
            no formal requirements
    \item you do not have to follow all house rules for the initial submission. This is only necessary once the manuscript is approved.
    \end{itemize}
  \end{itemize}
}

\frame{
\frametitle{Review}
%   \includegraphics[height=.2\textheight]{./path/to/graphicsfile}
  \begin{itemize}
    \item How could a more open reviewing process look like?
    \item Open Peer Review
    \begin{itemize}
      \item prepublication/postpublication
      \item invited/self-selected
      \item influences decision to accept Y/N
      \item review text openly available/hidden
    \end{itemize}
    \begin{itemize}
      \item Code/Workflow review
      \item If there are quantitative arguments in the paper, are data and computer code available so that interested readers can try to replicate and evaluate the findings?
      \item How to Share Data and Code when Submitting Papers to a Journal: Practical Questions \url{https://calc.hypotheses.org/2782}
    \end{itemize}
  \end{itemize}
}

\frame{
\frametitle{Revision}
%   \includegraphics[height=.2\textheight]{./path/to/graphicsfile}
\begin{itemize}
  \item So your manuscript is accept with major revisions. What next?
  \begin{itemize}
    \item  content revision: fine-tune argumentation
    \item  formal revision: make sure all guidelines are met and tables and figures look good
    \item  Overleaf.com allows for a separation of labour:
    \begin{itemize}
      \item author does the content revision
      \item student assistants take care of the menial tasks
      \item typesetters (yours truly) take care of tables, diagrams, and complicated issues.
    \end{itemize}
  \end{itemize}
\end{itemize}
}

\frame{
\frametitle{Proofreading}
%   \includegraphics[height=.2\textheight]{./path/to/graphicsfile}
  \begin{itemize}
    \item How could a more open proofreading/copyediting process look like?
    \item Get rid of publisherese and get in touch with ``real'' readers from different demographics
    \item Community Proofreading
    \item Each of the LangSci books is read by 10-20 proofreaders (1-2 chapters per person)
    \item All of them have different background
    \item Heterogeneous comments, but should be able to discover problems for different groups of readers.
  \end{itemize}
}

\frame{
\frametitle{Publication/Dissemination}
%   \includegraphics[height=.2\textheight]{./path/to/graphicsfile}

  \begin{itemize}
    \item  So the revised, typeset pdf is there. How would a more open approach to dissemination look like?
    \item no fees.
    \item easy download
    \item no vendor lock-in ``reading tools'' in the browser
    \item next to the pdf, the raw input material (numerical figures, graphics, computer code) is made available
    \item for people who prefer paper, print-on-demand is offered.
    \item Fair Open Access Principles: keep your copyright!
  \end{itemize}
}

\frame{
\frametitle{Reading}
%   \includegraphics[height=.2\textheight]{./path/to/graphicsfile}
\begin{itemize}
  \item How would a more open reading experience look like?
\end{itemize}

  \begin{itemize}
    \item  no fees for the reader
    \item  format neutral
    \begin{itemize}
      \item next to pdf, also other formats like HTML or epub are proposed
      \begin{itemize}
        \item LangSci does not offer this as of today
      \end{itemize}
    \end{itemize}

  \end{itemize}
}


\frame{
\frametitle{Iteration}
%   \includegraphics[height=.2\textheight]{./path/to/graphicsfile}
\begin{itemize}
  \item After publication, the readership will interact with the work. How can this be made more open?
\end{itemize}

  \begin{itemize}
    \item shorter iteration cycles
    \begin{itemize}
      \item make it easy to have subsequent editions to incorporate feedback
    \end{itemize}
    \item Post Publication Peer Review
    \begin{itemize}
      \item allow readers to easily leave comments about certain passages
      \item ``collaborative reading'' with PaperHive
    \end{itemize}
    \item ``forkability''
    \begin{itemize}
      \item allow others to ``rebuild'' your book based on the basic materials with amendments
    \end{itemize}
    \item use the Creative Commons Attribution license (CC-BY)
  \end{itemize}
}

\frame{
\frametitle{Prestige}
%   \includegraphics[height=.2\textheight]{./path/to/graphicsfile}
 \begin{itemize}
   \item some venues enjoy a lot of prestige, others don't, but the reasons for this are often historic and very often completely opaque
   \item open download figures can provide some transparency here
   \item but note that commodification/quantification is very often a bad proxy to judge scientific quality
 \end{itemize}
}

\frame{
\frametitle{Long term preservation}
%   \includegraphics[height=.2\textheight]{./path/to/graphicsfile}
  \begin{itemize}
    \item Special mechanisms have been devised to make sure that cease of operation of publishers does not entail that the works published by them are lost
    \item   CLOCKSS
    \item trigger events
    \begin{itemize}
      \item eg bankruptcy
    \end{itemize}
    \item for open publications: the trigger event is simply the initial publication. No need for contrived processes like CLOCKSS
  \end{itemize}
}






%
% \section{Technical aspects}
% \frame{
% \frametitle{Components of book production}
% %   \includegraphics[height=.2\textheight]{./path/to/graphicsfile}
%   \begin{itemize}
%     \item  Raw data (recordings, surveys, samples, manuscripts)
%     \item  Code (scripts in R, python, praat)
%     \item Prose (The text surrounding your data)
%     \item Bibliography
%     \item Proportions of these may vary
%     \item Do keep them separate and well-organized
%     \begin{itemize}
%       \item do use a references manager like Zotero
%     \end{itemize}
%   \end{itemize}
% }
%
% \frame{
% \frametitle{Versioning}
% %   \includegraphics[height=.2\textheight]{./path/to/graphicsfile}
%   \begin{itemize}
%     \item Versioning is crucial for large projects
%     \item Back-up
%     \item Roll-back
%     \item Fork
%     \item Collaborate
%     \item Merge
%     \item Automated tests
%     \item Get feedback
%   \end{itemize}
% }
%
% \frame{
% \frametitle{Versioning: history}
%   \includegraphics[height=\textheight]{juskancommits.png}
% }
%
%
% \frame{
% \frametitle{Versioning: diffs}
%   \includegraphics[height=\textheight]{juskansplit.png}
% }
%
%
% \frame{
% \frametitle{Versioning: integration}
%   \includegraphics[height=\textheight]{bodowinter.png}
%   }
%
% \frame{
% \frametitle{GitHub}
%    \includegraphics[height=\textheight]{github.png}
% }
%
% \frame{
% \frametitle{GitHub}
%    \includegraphics[height=\textheight]{githubjapanese.png}
% }
%
% \frame{
% \frametitle{Versioning and neat project structure}
% \begin{itemize}
%   \item Fire-and-forget graphic production is problematic
%   \item adjustments to colours or resolution are not possible anymore at a later point in time.
% \end{itemize}
%
%   \includegraphics[height=\textheight]{pngR.png}\pause~\includegraphics[height=\textheight]{pngR2.png}
% }
%
%
%
% \frame{
% \frametitle{Versioning and neat project structure}
% \begin{columns}
% \begin{column}{5cm}
% \includegraphics[width=1\textwidth]{folderstructure.png}
% \end{column}
% \begin{column}{5cm}
% \vspace*{-3cm}
% \includegraphics[width=\textwidth]{NitzkecsvLO.png}\\
% \includegraphics[width=1.2\textwidth]{nitzkebars.png}
% \end{column}
% \end{columns}
% }
%
%
%
% % \frame{
% % \frametitle{Versioning: GitHub}
% %   \includegraphics[height=\textheight]{unicodecookbook.png}
% % }
% %
% % \frame{
% % \frametitle{Versioning: releases}
% %   \includegraphics[height=\textheight]{releases.png}
% % }
%
%
% \frame{
% \frametitle{Repositories}
% %   \includegraphics[height=.2\textheight]{./path/to/graphicsfile}
%   \begin{itemize}
%     \item Repositories collect documents and research data
%       \begin{itemize}
%         \item \textbf{discipline-specific} or \textbf{general purpose}
%         \item \textbf{preprint} or \textbf{postprint}
%         \item \textbf{text} or \textbf{data}
%       \end{itemize}
%   \end{itemize}
% }
%
% \frame{
% \frametitle{figshare}
%    \includegraphics[height=\textheight]{figshare.png}
% }
%
% % \frame{
% %    \includegraphics[height=\textheight]{dryad.png}
% % }
%
% \frame{
% \frametitle{Trolling}
%    \includegraphics[height=\textheight]{trolling.png}
% }
%
% \frame{
% \frametitle{LingBuzz}
%    \includegraphics[height=\textheight]{lingbuzz.png}
% }
%
% \frame{
% \frametitle{semanticsarchive.net}
%    \includegraphics[height=\textheight]{semanticsarchive.png}
% }
%
% \frame{
% \frametitle{Rutgers Optimality Archive}
%    \includegraphics[height=\textheight]{roa.png}
% }
%
%
% \frame{
% \frametitle{Endangered Languages\\ Archive}
%    \includegraphics[height=\textheight]{elar.png}
% }
%
%
% \frame{
% \frametitle{Zenodo}
%    \includegraphics[height=\textheight]{zenodo.png}
% }
%
%
%
% \frame{
% \frametitle{Academia and Researchgate}
%   \begin{itemize}
%     \item  Academia.edu and researchgate are NOT repositories
%     \item both make money from restricting access
%     \end{itemize}
% }
%
% \frame{
% \frametitle{A note on scooping}
%   \begin{itemize}
%     \item  Scooping means that someone ``steals'' your research while you are still finalising it
%     \item that fear is by and large unwarranted
%     \item you can count yourself happy if anybody IS actually interested in your data!
%     \item most research actually struggles a lot more with a lack of interest to outsiders rather than scooping
%     \item preprint servers allow you to register your data and establish primacy
%   \end{itemize}
% }
%
%
% \frame{
% \frametitle{Recommendations for\\ technical sustainability}
%   \begin{itemize}
%   \highlight
%     \item conceptually separate your data, code, and prose text
%     \item make sure you can replicate all steps of your analysis
%     \item ideally, specify automated pipelines for your analysis
%     \item do not hardwire data and code
%     \item use a citation manager
%     \item use a versioning tool
%     \begin{itemize}\highlight
%       \item local or university or cloud
%     \end{itemize}
%     \item use a preprint server
%     \item backup\pause
%     \item backup\pause
%     \item backup
%     \end{itemize}
% %   \color{black}\normalfont
% }
%
%
%
% \section{Legal aspects}
%
% \frame{
% \frametitle{Legal sustainability}
%   \includegraphics[height=\textheight]{justitia.png}
% }
%
% \frame{
% \frametitle{Intellectual Property Rights}
% %   \includegraphics[height=.2\textheight]{./path/to/graphicsfile}
%   \begin{itemize}
%     \item  ``Copyright'' in the Anglo-Saxon countries
%     \item  ``Urheberrecht'' in continental Europe
%     \item These are different!
%     \item Once you create somehting, you can decide who can use it and how
%     \item Choose wisely and be explicit!
%   \end{itemize}
% }
%
% \frame{
% \frametitle{License}
% %   \includegraphics[height=.2\textheight]{./path/to/graphicsfile}
%   \begin{itemize}
%     \item  Usage rights can be differentiated as follows
%     \begin{itemize}
%       \item geographical restriction (e.g. \textit{only for Germany})
%       \item type restrictions (e.g. \textit{only for print})
%       \item exclusive (no-one else has the right) vs. non-exclusive (other people may also get the rights)
%       \item copyright transfer agreement (Anglo-Saxon culture)
%       \item total buy-out
%     \end{itemize}
%   \end{itemize}
% }
%
% \frame{
% \frametitle{Copyright transfer agreement}
% \includegraphics[width=\textwidth]{copyrighttransfer.pdf}
% }
%
% \frame{
% \frametitle{Contracts}
% %   \includegraphics[height=.2\textheight]{./path/to/graphicsfile}
%   \begin{itemize}
%     \item  Publisher contracts restrict everybody, including yourself
%     \item once you sign away your copyright, you have no longer the right to use your own material
%     \item to reuse tables or graphics in subsequent works of yours, you must first ask the new rights holder for permission
%     \item chasing rights is incredibly annoying
%     \item publishers will put your content behind a paywall, meaning that it can actually be more difficult to access once it is officially published than before
%     \item publisher vary as to whether and when they allow books to be hosted in repositories
%   \end{itemize}
%     }
%
% \frame{
% \frametitle{Availability of Nordhoff (ed.) (2012)}
% \includegraphics[width=\textwidth]{brillnordhoff.png}
%  }
%
% \frame{
% \frametitle{Recommendations for legal sustainability}
%   \begin{itemize}
%   \highlight
%     \item Publish your works under a Creative Commons license, preferably CC-BY
%     \begin{itemize}\highlight
%       \item see the Berlin|Budapest|Bethesda declarations on Open Access
%     \end{itemize}
%     \item This license clearly states who the copyright holder is and how the content can be reused (typically requiring only the attribution of the original author)
%     \item see Fair Open Access principles on \url{https://www.lingoa.eu/about/mission}
%     \item do not sign away your copyright \pause
%     \item do not sign away your copyright \pause
%     \item do not sign away your copyright
%   \end{itemize}
% }
%
% %                 The editor hereby assigns to the Publisher the full copyright in the Work [...]. Consequently, the Publisher shall have the exclusive right \textbf{throughout the world} to publish and sell th Work in \textbf{all languages}, in whole or in part, including, \textbf{without limitation}, \textbf{any} abridgement and substantial part thereof, in book form and in \textbf{any other form} including, without limitation, mechanical, digital electronic, and visual reproduction, electronic storage and retrieval systems, including internet and intranet delivery and \textbf{all other forms of electronic publication now known or hereinafter invented}.
%
% \section{Financial aspects}
% \frame{
% \frametitle{Financial aspects of\\ sustainability}
%   \includegraphics[height=\textheight]{calculation.jpg}
%  {\tiny CC-BY-ND Dennis Skley \url{https://www.flickr.com/photos/dskley/14895278831}}
%  }
%
% \frame{
% \frametitle{Print run}
% %   \includegraphics[height=.2\textheight]{./path/to/graphicsfile}
%   \begin{itemize}
%     \item  How many books does a publisher print?
%   \end{itemize}
% }
%
% \frame{
% \frametitle{Cost for printing one book per Print-on-Demand}
% \includegraphics[width=\textwidth]{printingcosts.png}
% }
%
% \frame{
% \frametitle{Print run}
% %   \includegraphics[height=.2\textheight]{./path/to/graphicsfile}
%   \begin{itemize}
%     \item  How many books does a publisher sell?
%     \item How much does the author typically earn?
%   \end{itemize}
% }
%
% \frame{
%   \includegraphics[width=\textwidth]{royalty.pdf}
% }
%
% \frame{
% \frametitle{Funding models: reader pays}
% \includegraphics[height=.8\textheight]{brillnordhoff.png}
%   \begin{itemize}
%     \item  item-based one-off purchase
%     \item  institutional bundle
%     \item  patron driven acquisition
%   \end{itemize}
% }
%
% \frame{
% \frametitle{Services provided by\\ publishers}
% %   \includegraphics[height=.2\textheight]{./path/to/graphicsfile}
%   \begin{itemize}
%     \item review facilitation (or not)
%     \item  language polishing (or not)
%     \item  copy-editing (or not)
%     \item  typesetting (or not)
%     \item  indexing (or not)
%   \end{itemize}
% }
%
% \frame{
% \frametitle{Funding models: author pays}
% %   \includegraphics[height=.2\textheight]{./path/to/graphicsfile}
%   \begin{itemize}
%     \item  Article processing charges (APCs)
%     \item  Book processing charges (BPCs)
%     \item Book chapter processing charges (BCPCs)
%     \item What is the cost of publishing an article? \pause
%     \item What is the cost of publishing a book? \pause
%     \item \textit{Scielo} (Brazil) charges 90 USD for one article
%     \item \textit{Nature Communications} charges 5200 USD for one article
%     \item this explains the profit margins of 30-40\% of the large publishers (Springer, Wiley, Elsevier)
%     \item book costs are estimated between 3\,500 and 130\,000 EUR
%   \end{itemize}
% }
%
% \frame{
% \frametitle{Funding model: discipline\\ pays}
% %   \includegraphics[height=.2\textheight]{./path/to/graphicsfile}
%   \begin{itemize}
%     \item institutional funding
%     \begin{itemize}
%       \item Uni Bern: \url{https://bop.unibe.ch/linguistik-online}
%       \item Uni Hawaii: Journal of Language Documentation and Conservation
%       \item Uni Dartmouth: Linguistic Discovery
%       \item MIT/LSA: Semantics and Pragmatics
%     \end{itemize}
%     \item no fees for readers, no fees for authors
%     \item host institution covers the costs
%   \end{itemize}
% }
%
% \frame{
% \frametitle{Funding model:\\ decentralized funding}
% %   \includegraphics[height=.2\textheight]{./path/to/graphicsfile}
%   \begin{itemize}
%     \item  Open Library of Humanities (300+ supporting institutions)
%     \begin{itemize}
%       \item Glossa
%       \item Journal of Portuguese Linguistics
%       \item Laboratory Phonology
%       \item Italian Journal of Linguistics
%     \end{itemize}
%     \item Language Science Press (100+ supporters)
%     \item Open Library Politikwissenschaft (40+ supporting institutions)
%   \end{itemize}
% }
%
% \frame{
% \frametitle{More figures and calculations}
% \includegraphics[height=\textheight]{cookbook.png}
% \includegraphics[height=\textheight]{businessmodel.png}
% }
%
% \frame{
% \frametitle{Financial aspects:\\ recommendations}
% %   \includegraphics[height=.2\textheight]{./path/to/graphicsfile}
%   \begin{itemize}
%   \highlight
%     \item go for discipline-pays
%     \begin{itemize}\highlight
%     \item else go for author-pays if funds are available
%       \begin{itemize}\highlight
%           \item else go for reader-pays and deposit preprints/postprints in repositories.
%       \end{itemize}
%     \end{itemize}
%   \end{itemize}
% }
%
% \section{Sociological aspects}
% \frame{
% \frametitle{Sociological aspects}
% \hspace*{-3cm}\includegraphics[width=1.6\textwidth]{network.pdf}
% }
%
% \frame{
% \frametitle{Functions of a publisher}
% %   \includegraphics[height=.2\textheight]{./path/to/graphicsfile}
%   \begin{enumerate}
%     \item  registration \only<2->{$\to$\hfill preprint server}
%     \item   certification \only<3->{$\to$\hfill ???}
%     \item   dissemination  \only<4->{$\to$\hfill repository, social media, print-on-demand}
%     \item   archiving    \only<5->{$\to$\hfill Zenodo}
%   \end{enumerate}
% \uncover<6->{The main function which is difficult to substitute remains certification of quality.}
% }
%
% \frame{
% \frametitle{Dissemination}
% \includegraphics[height=\textheight]{facebook.png}
% }
%
% \frame{
% \frametitle{Dissemination}
% \includegraphics[width=\textwidth]{cumulativeall10000.png}
% }
%
% \frame{
% \frametitle{Outsourcing of research\\ evaluation}
% %   \includegraphics[height=.2\textheight]{./path/to/graphicsfile}
%
%   \parbox{1cm}{~}\begin{itemize}
%     \item Universities want to hire good researchers
%     \item The amount of research produced by all candidates combined is too much for the committee to read
%     \item The committee use some proxy to judge quality: the brand of the publisher or the journal
%     \begin{itemize}
%       \item good publishers, regular publishers, bad publishers, ...
%     \end{itemize}
%     \item NB: quality selection is done by the (unpaid) editors, not by the publishers
%     \item the quality selection process is more or less the same for all publishers
%   \end{itemize}
% }
%
% \frame{
% \frametitle{Brands}
% %   \includegraphics[height=.2\textheight]{./path/to/graphicsfile}
%   \begin{itemize}
%     \item Once a publisher is perceived as prestigious, a self-reinforcing loop starts:
%     \begin{enumerate}
%       \item researchers present better manuscripts to the publisher
%       \item the publisher has more material to choose from
%       \item the average quality of the publisher gets better
%       \item prestige goes up
%       \item start over
%     \end{enumerate}
%     \item Crucially, there is not necessarily more work being done by prestigious publishers
%     \item But since the brand is owned by a company, they increase the prices beyond proportion
%   \end{itemize}
% }
%
% \frame{
% \frametitle{Proofs by a prestigious\\ publisher}
%   \includegraphics[width=.6\textwidth]{dg.png}
%   ~~~~
%   \vspace*{-7cm}\includegraphics[width=.3\textwidth]{semf.png}\vspace*{7cm}  }
%
% \frame{
% \frametitle{Errors and Journal Impact Factor}
% \includegraphics[width=\textwidth]{jif.png}
% \url{biorxiv.org/content/biorxiv/early/2016/08/25/071530.full.pdf}
% }
%
% \frame{
% \frametitle{It's the employability, stupid}
% %   \includegraphics[height=.2\textheight]{./path/to/graphicsfile}
%   \begin{itemize}
%     \item Being employed as a professor between 40 and 65 years gives you 3\,000\,000 EUR income.
%     \item It is economically rational for each individual researcher to spend 5\,000+ EUR for an article in \textit{Nature} if this improves your chances of becoming a professor
%     \item It is economically rational for \textit{Nature} to charge as much as they can since they are a public company
%     \item But this sucks money out of research, which could better be used to improve teaching, fund postdocs or develop vaccines.
%     \item Two solutions:
%     \begin{itemize}
%       \item decouple research evaluation from the place of publication (DORA)
%       \item move the brands from the for-profit-sphere to the non-profit-sphere (Fair OA)
%     \end{itemize}
%   \end{itemize}
% }
%
% \frame{
% \frametitle{Choice of publisher according to JS Caux}
% %   \includegraphics[height=.2\textheight]{./path/to/graphicsfile}
%   \begin{description}
%     \item[\textbf{Gold}] Content is free for readers
%     \item[\textbf{Platinum}] as above, and free for authors
%     \item[\textbf{Palladium}] as above, and all financial statements are disclosed\vspace*{.3cm}\\\pause
%     \item[\textbf{Iron}] pay-to-read\pause
%     \item[\textbf{Lead}] editorial and financial aspects are not hermetically decoupled
%   \end{description}\pause
%   \url{https://jscaux.org/blog/post/2017/09/20/noble-metals-noble-cause/}
%   }
%
%
% \frame{
% \frametitle{Recommendations for\\ financial aspects}
% %   \includegraphics[height=.2\textheight]{./path/to/graphicsfile}
%   \begin{itemize}
%   \highlight
%     \item choose a scholar-led community-owned publisher
%     \item             do not exacerbate libraries' burden
%    \item          lobby against stupid criteria of research assesement like the Journal Impact Factor
%   \end{itemize}
% }
%
%
%
%
% % \section{Political aspects}
% % \frame{
% % \frametitle{Political aspects}
% % %   \includegraphics[height=.2\textheight]{./path/to/graphicsfile}
% % }
%
% % \frame{
% % \frametitle{Political initiatives}
% % %   \includegraphics[height=.2\textheight]{./path/to/graphicsfile}
% %   \begin{itemize}
% %     \item  DEAL: Germany negotiates national OA licensing programmes with big publishers
% %     \item  Plan S: Research funders will require 100\% Open Access from projects they fund
% %     \begin{itemize}
% %     \item There are indications that both actions probably lead to an increase in authory-pays publishing fees
% %     \end{itemize}
% %     \item Sci-Hub and LibGen allow you to access paywalled content and are probably more successful an instrument than all political initiatives
% %   \end{itemize}
% % }
% %
% % \frame{
% % \frametitle{Pollitical miscellanea}
% % %   \includegraphics[height=.2\textheight]{./path/to/graphicsfile}
% %   \begin{itemize}
% %     \item  Swiss National Fund pays generous processing charges
% %     \item Austrian FWF is more on the repository side
% %     \item The San Francisco Declaration on Research Assessment (DORA) calls for a move away from journals/publishers as proxies for research evaluation.
% %   \end{itemize}
% % }
%
%
% \frame{
% \frametitle{Wrap-up: recommendations\\ \mbox{for sustainable book publications}}
% %   \includegraphics[height=.2\textheight]{./path/to/graphicsfile}
%   \begin{itemize}
%   \highlight
%     \item separate data, code, and text
%     \item use a versioning system
%     \item backup\\~
%     \item publish all your data in appropriate repositories
%     \item use preprint servers\\~
%     \item use the CC-0 license for data and the CC-BY license for text
%     \item do not sign away your copyright\\~
%     \item care for your discipline, go for scholar-led community-owned publishers
%   \end{itemize}
% }
%
% \frame{
% \frametitle{Thank you}
%   \includegraphics[width=\textwidth]{nebrija.jpg}
% }

\end{document}
